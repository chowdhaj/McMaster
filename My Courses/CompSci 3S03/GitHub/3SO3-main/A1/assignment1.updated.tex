\documentclass{article}
\usepackage[utf8]{inputenc}
\usepackage[T1]{fontenc}
\usepackage[margin=0.7in]{geometry}
\usepackage{graphicx}
\usepackage{amsmath}
\usepackage{amssymb}
\usepackage{xcolor}
\usepackage{color}
\usepackage{listings}
%\usepackage{minted}
\usepackage{listings}
\usepackage{syntax}

\title{3S03 - Software Testing \\
Assignment 1}
\author{Mark Hutchison \\
hutchm6@mcmaster.ca \and
Jatin Chowdhary \\
chowdhaj@mcmaster.ca}
\date{\today}

\begin{document}

\maketitle

\tableofcontents

\section{Question 1 - The Dangers of Inheritance from a Testing
Viewpoint}

Inheritance is an extremely useful concept in object-oriented
programming; it allows developers to save time by maximizing code
reusability. However, from a testing viewpoint, the use of an
inheritance hierarchy can have detrimental effects on the
controllability and observability of code. The following is a simple
yet effective example that demonstrates this. In addition, the
following code utilizes proper modular code practices; file
separation, to name a few.

\begin{lstlisting}[language=Python]
    class Cat(object):

        def __init__(self, colour: str, size: str, home: str):
            self.colour: str = colour
            self.size: str = size
            self.home: str = home

        def __str__(self) -> str:
            if self.size == "small":
                return "Meow"
            elif self.size == "medium":
                return "Mow"
            elif self.size == "big":
                return "Roar"
            else:
                return "Hiss"

        def play(self, duration: float, toy_weight: float) -> bool:
            if toy_weight > 20:
                raise ToyToHeavy()
            if duration > 1440:
                raise DayWasted()
            return duration < toy_weight

    class Lion(Cat):

        def __init__(self, colour: str):
            super().__init__(colour, "big", "Africa")
        def __str__(self)__ -> str:
            return super().__str__()
        def play(self, duration: float, toy_weight: float) -> bool:
            return super().play(duration, toy_weight)
\end{lstlisting}

% \begin{minted}{python}
% class Cat(object):
%         def __init__(self, colour: str, size: str, home: str):
%             self.colour: str = colour
%             self.size: str = size
%             self.home: str = home
%         def __str__(self) -> str:
%             if self.size == "small":
%                 return "Meow"
%             elif self.size == "medium":
%                 return "Mow"
%             elif self.size == "big":
%                 return "Roar"
%             else:
%                 return "Hiss"
%         def play(self, duration: float, toy_weight: float) -> bool:
%             if toy_weight > 20:
%                 raise ToyToHeavy()
%             if duration > 1440:
%                 raise DayWasted()
%             return duration < toy_weight
%
%     class Lion(Cat):
%         def __init__(self, colour: str):
%             super().__init__(colour, "big", "Africa")
%         def __str__(self)__ -> str:
%             return super().__str__()
%         def play(self, duration: float, toy_weight: float) -> bool:
%             return super().play(duration, toy_weight)
% \end{minted}

Although this example is relatively small, the following issues
(\textbf{Observability} and /textbf{Controllability}) are present:

\subsection{Observability}

From an observability perspective, the example above lacks
observability; it is not easy to observe the behaviour of the program.
For example, in order to understand the behaviour of the
\texttt{Lion().\_\_str\_\_()} function, the entire
\texttt{Cat().\_\_str\_\_()} function, in the parent class, needs to
be examined as well. Upon examining the function in the parent class,
the variables of the \texttt{Lion} class instance need to be examined
as well. Evidently, observability has been compromised. Our initial
goal of deducing the result of the \texttt{Lion().\_\_str\_\_()}
function has turned into a tedious analysis of 2 classes,
\texttt{Lion} and \texttt{Cat}, and requires tracing variables from
one to the other. Although this example is relatively small, repeating
this on a much larger scale - tracing multiple functions across
multiple parent classes to see the effects of each function - hinders
observability. Furthermore, Python allows multiple parent classes to
be passed in to a single child, if establishing an inheritance chain
is not required. However, it can be argued that inheritance chaining
is the worst option in terms of observability due to the recursive
search of the parent functions. From a testing perspective, this is a
nightmare.
%% This is 9 sentences; technically it is a paragraph.

\subsection{Controllability}

In terms of controllability, inheritance requires developers to not
only worry about how the function parameters for the current function
call, but also how those parameters will climb the inheritance tree.
The \texttt{play()} function demonstrates this. Although the code in
the \texttt{Lion} class does not show any errors, there are 2
conditions in the parent class (i.e. \texttt{Cat}) that can cause
errors. As previously stated, developers need to concern themselves
with how the current function uses the parameters, in addition to how
the parent class uses the parameters; controllability has been
compromised. Furthermore, not only does the child class need to be
verified as correct, but the parent class as well. Failure to do so
can cause errors to propagate from the parent class to the child
class.
%% This is 6 sentences; definitely a paragraph.

\section{Question 2 - Junit on the \texttt{sort()} function}

Now, this question is more flawed in the theory of what the test does,
since the implementation of \texttt{sort()} is not known. However, if
there is a fault this test case would not find it:

\begin{itemize}

    \item \textbf{Reachability:} Running the \texttt{sort()} function
        does traverse us to a potential final state that contains the
        fault. Again, we cannot confirm alone that a fault exists, but
        we are moving into a new state here, and it is our final state
        reached by the test case.
    \item \textbf{Infection:} The final state of sorted could be out
        of order, thus being "infected" by a fault. Our test case, run
        likely by a test oracle, needs to validate the ENTIRE order of
        the list to ensure no fault exists.
    \item \textbf{Propagation:} The potentially infected state is the
        final state as an assertion takes place immediately after our
        sorting occurs. This means that the fault does indeed have a
        chance to cause a fatal error or unexpected result.
    \item \textbf{Revealability:} The assert statement on reveals that
        the first item is correct, leaving 3 items to be potentially
        unsorted. We, however, cannot see these items and thus cannot
        see the fault if it exists. We must see and traverse the
        entire sorted list to be sure there is no fault propagating in
        this test. Again, this is likely being handled by some sort of
        test oracle, in which would attempt to force you to review the
        list contents in its entirety, not just 1 item.

\end{itemize}

Again, no fault might actually exist, but the test provided does not
do enough to say that. As of this moment, this test can contain a
fault and allow it to pass without issue.

\section{Question 3 - Test Driven Development}

\subsection{Requirements Analysis}

The first issue encountered when preparing to design test cases for
this problem is the lack of specificity; no actual requirements are
given. Apart from the names of each function, the expected type
signature and functionality of each function is not known. Without the
ability to concretely determine these requirements, assumptions are
going to be made about the functions and their behavior; this creates
ambiguity - discussed later.

\medskip

When coming up with criteria without requirements, we need to think
about overall project cost, and the repercussions of our assumptions
being wrong. Due to this, minimizing cost/work is essential. The
assumptions we make need to reduce the amount of effort/work we put
into the (\texttt{Calc}) functions. Hypothetically, if a client states
further requirements, then it is easier to add more functionality
and negotiate extra pay than it is to waste time on work the client
did not want; and bare the cost that comes with it.

\medskip

The three major requirements not detailed in the specification
document are:

\begin{itemize}
    \item All 4 \texttt{Calc} functions will be implemented on
        integers, and return integers. Hence, there will be no issues
        that arise with floating point arithmetic. Furthermore, we
        will follow conventions that come with integer division, such
        as flooring floating point results.
    \item In the event of $overflow^{*}$, depending on how Java handles
        it, we need to do something specific. My original thought was
        to allow any errors to permeate through and be caught later in
        exception handling, however, Java may allow overflow.
    \item The lack of explicit details about the intended scalability
        of this project in the future. We do not know how this class
        is intended to interact with others, so we must keep
        minimalism in mind. We can achieve this by using correct
        design principles that anticipate change.
\end{itemize}

\subsection{Pre-Coding Test Generation}

Coming up with tests prior to implementing the code was potentially
one of the harder things to do, even with functions so easy. Of course
you have your standard tests like just doing
\textit{$3 \cdot 4 == 12$} or \textit{$5 + 8 == 13$}, but then you
also have to account for edge cases that come from properties related
to domains. Some examples include:

\begin{itemize}

    \item \textbf{Identity Cases:} Nearly every operator has two
        identity cases: The zero case and the ``do nothing'' case.

        \begin{itemize}
            \item Addition: $n + (-n) == 0$ and  $n + 0 == n$
            \item Subtraction: $n - n == 0$ and $n - 0 == n$
            \item Multiplication: $n \cdot 0 == 0$ and $n \cdot
                1 == n$
            \item Division: $\frac{n}{n} == 1$ and $\frac{n}{1} == n$
        \end{itemize}

    \item \textbf{Equivalency Cases:} Sometimes, these operators
        become each other. Since \textit{a} and \textit{b} must be
        integers, we do not worry about Multiplication becoming
        Division or Division becoming Multiplication. However,
        subtracting a negative value is just adding the positive,
        while adding a negative value is just subtracting the
        positive. We can add automated testing for these cases.

    \item \textbf{Boundary tests:} There are some boundary tests that
    will need to be implemented:

        \begin{itemize}
            \item Addition: You cannot add anything to the maximum
                integer, so we expect an error here. Also, attempting
                to add a negative value to the minimum integer is also
                an error. \textit{Note: Our initial idea for boundary
                testing on addition is to check if the result ends up
                being negative, because 2 positive integers must add
                up to a positive integer. If the result is negative,
                then $overflow^*$ has occurred. However, this idea is
                not sound because if the result overflows to a
                positive number, then an error will not be raised,
                despite the occurrence of an overflow. Similar logic
                applies to subtraction.}
            \item Subtraction: You cannot subtract a negative value
                from the minimum integer, so we expect an error here.
                \textit{Note: Boundary testing for subtraction poses
                similar problems outlined above, under Addition.}
            \item Multiplication: The absolute value of the product of
                two integers cannot exceed the integer boundaries.
            \item Division: You cannot divide by 0.
        \end{itemize}

\end{itemize}

Once we implement all of the specific properties mentioned in a test
case, add some standard unit tests to verify that specific values are
equal to what we expect, and verify we are happy with the testing
suite - We can move on to the code.

\medskip

Coding these are all relatively simple; each operator is only one
line, after all! However, I wanted to make it explicitly clear that
our code can error from performing specific operations, as those
errors exist in our test cases. I am not going to do any if statements
or anything like that, I am simply going to allow the errors through
to the test case and denote the function can throw the expected error.

\subsection{Post-Coding Modifications}

As feared, Java does allow $overflow^*$ (and $underflow$). Due to
this, a decision needs to be made: Either explicitly account for it in
all 4 methods (including the given source code), or refactor the test
cases to allow the flow edge cases. The former requires more work, but
the latter removes a good chunk of our boundary tests, so both are a
costly choice. In the end, I refer to the minimalism point brought up
in the discussion about requirements: Imposing a harder to test
condition on the code is likely a worse option than removing the test
cases. If the overflowed is not desired, we can account for it after
acquiring more information.

\newpage

\textit{$^*$Note: The terms overflow and underflow follow normal definitions laid out by the principles in computer science. If a number exceeds the maximum, then overflow has occured. If a number goes below the minimum, then overflow has (also) occured. Our definitions for overflow are based on modular/clockwork arithmetic. Examples are: \texttt{MAX_INT + 1} and \texttt{MIN_INT - 1}, respectively. The term underflow refers to a number that is too small to accurately represent, so the value ends up being 0. Since we are only dealing with operations on integers, we can ignore underflow and only focus on overflow.}

\end{document}
