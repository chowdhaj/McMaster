\documentclass{article}
\usepackage[utf8]{inputenc}
\usepackage[T1]{fontenc}
\usepackage[margin=0.7in]{geometry}
\usepackage{graphicx}
\usepackage{amsmath}
\usepackage{amssymb}
\usepackage{xcolor}
\usepackage{color}
\usepackage{listings}
%\usepackage{minted}
\usepackage{listings}
\usepackage{syntax}

\title{3S03 - Software Testing \\
Assignment 1}
\author{Mark Hutchison \\
hutchm6@mcmaster.ca \and
Jatin Chowdhary \\
chowdhaj@mcmaster.ca}
\date{\today}

\begin{document}

\maketitle

\tableofcontents

\section{Question 1 - The Dangers of Inheritance from a Testing
Viewpoint}

Yes, inheritance can be great for coders, but there are always going
to be issues that come with this benefit. For the purposes of this
example, let's use a simple example that also uses proper modular code
practices like file separation and so on:

\begin{lstlisting}[language=Python]
    class Cat(object):

        def __init__(self, colour: str, size: str, home: str):
            self.colour: str = colour
            self.size: str = size
            self.home: str = home

        def __str__(self) -> str:
            if self.size == "small":
                return "Meow"
            elif self.size == "medium":
                return "Mow"
            elif self.size == "big":
                return "Roar"
            else:
                return "Hiss"

        def play(self, duration: float, toy_weight: float) -> bool:
            if toy_weight > 20:
                raise ToyToHeavy()
            if duration > 1440:
                raise DayWasted()
            return duration < toy_weight

    class Lion(Cat):

        def __init__(self, colour: str):
            super().__init__(colour, "big", "Africa")
        def __str__(self)__ -> str:
            return super().__str__()
        def play(self, duration: float, toy_weight: float) -> bool:
            return super().play(duration, toy_weight)
\end{lstlisting}

% \begin{minted}{python}
% class Cat(object):
%         def __init__(self, colour: str, size: str, home: str):
%             self.colour: str = colour
%             self.size: str = size
%             self.home: str = home
%         def __str__(self) -> str:
%             if self.size == "small":
%                 return "Meow"
%             elif self.size == "medium":
%                 return "Mow"
%             elif self.size == "big":
%                 return "Roar"
%             else:
%                 return "Hiss"
%         def play(self, duration: float, toy_weight: float) -> bool:
%             if toy_weight > 20:
%                 raise ToyToHeavy()
%             if duration > 1440:
%                 raise DayWasted()
%             return duration < toy_weight
%
%     class Lion(Cat):
%         def __init__(self, colour: str):
%             super().__init__(colour, "big", "Africa")
%         def __str__(self)__ -> str:
%             return super().__str__()
%         def play(self, duration: float, toy_weight: float) -> bool:
%             return super().play(duration, toy_weight)
% \end{minted}

Now, although this example is small scale, we can still see some
obvious issues:

\subsection{Observability}

In order to know what \texttt{Lion().\_\_str\_\_()} will do, we need
to examine the entirety of the \texttt{Cat().\_\_str\_\_()}. When
examining the function of the parent class, we will then notice that
we need to check the variables of the \texttt{Lion} class instance.
Now, what should have been a rather straightforward string result has
become the analysis of 2 classes and tracing variables from one to the
other. Imagine this on a much grander scale; having to trace multiple
functions across multiple parent classes to see the effects of one
versus the other. Python even allows multiple parent classes to be
passed in to a single child if you don't want to establish an
inheritance chain. However, I would argue inheritance chaining is the
worse option in terms of observability due to the recursive search of
the parent functions.

\subsection{Controllability}

In terms of controllability, you need to not only worry about how the
function parameters for the current function call, but how those
parameters will climb the inheritance tree. Look at the
\texttt{play()} function; although the code of the \texttt{Lion} class
doesn't show the errors, there are 2 conditions in the parent class
that can cause errors. You need to worry about how the parent uses
these values as well as how your current function uses them.
Additionally, your parent function now needs to be verified as correct
as your child class, or else errors from the parent will propagate to
the child!

\section{Question 2 - Junit on the \texttt{sort()} function}

Now, this question is more flawed in the theory of what the test does,
seeing as we can't see how \texttt{sort()} was implemented. If there
is a fault however, this test case would not find it:

\begin{itemize}

    \item \textbf{Reachability:} Running the \texttt{sort()} function
        does traverse us to a potential final state that contains the
        fault. Again, we can't confirm alone that a fault exists, but
        we are moving into a new state here, and it is our final state
        reached by the test case.
    \item \textbf{Infection:} The final state of sorted could be out
        of order, thus being "infected" by a fault. Our test case, run
        likely by a test oracle, needs to validate the ENTIRE order of
        the list to ensure no fault exists.
    \item \textbf{Propagation:} The potentially infected state is the
        final state as an assertion takes place immediately after our
        sorting occurs. This means that the fault does indeed have a
        chance to cause a fatal error or unexpected result.
    \item \textbf{Revealability:} The assert statement on reveals that
        the first item is correct, leaving 3 items to be potentially
        unsorted. We, however, cannot see these items and thus cannot
        see the fault if it exists. We must see and traverse the
        entire sorted list to be sure there is no fault propagating in
        this test. Again, this is likely being handled by some sort of
        test oracle, in which would attempt to force you to review the
        list contents in its entirety, not just 1 item.

\end{itemize}

Again, no fault might actually exist, but the test provided doesn't do
enough to say that. As of this moment, this test can contain a fault
and allow it to pass without issue.

\section{Question 3 - Test Driven Development}

\subsection{Requirements Analysis}

The first issue encountered when preparing to design test cases for this problem is the lack of actual requirements given. Apart from the names of each function, the expected type signature and functionality of each function is not known. Without the ability to concretely determine these requirements, assumptions are going to be made about the functions and their behavior.

When coming up with criteria without requirements, we need to think about overall project cost, and the repercussions of our assumptions being wrong. Because of this, to minimize cost, the assumptions we make need to be in order to reduce the amount of effort or work put into these functions. If they need to impose further specifications on the work, it is easier to add more and negotiate the extra pay than it is to waste the time on work they didn't want and eat the cost that comes with it.

The three major requirements not detailed in the requirements document are:

\begin{itemize}
    \item All 4 Calc functions will be implemented on integers, and return integers. So there will be no issues with floating point arithmetic, and we will follow conventions that come with integer division, such as flooring floating point results.
    \item In the event of overflow or underflow, depending on how Java handles it, we need to do something specific. My original thought was to allow any errors to permeate through and be caught later in exception handling, but Java may allow the overflow.
    \item The lack of explicit details about the intended scalability of this project in the future. We don't know how this class is intended to interact with others, so we must keep minimalism in mind.
\end{itemize}

\subsection{Pre-Coding Test Generation}

Coming up with tests prior to implementing the code was potentially
one of the harder things to do, even with functions so easy. Of course
you have your standard tests like just doing $3 \cdot 4 == 12$ or
$5 + 8 == 13$, but then you also have to account for edge cases that
come from properties related to domains. Some examples include:

\begin{itemize}

    \item \textbf{Identity Cases:} Nearly every operator has two
        identity cases: The zero case and the ``do nothing'' case.

        \begin{itemize}
            \item Addition: $n + (-n) == 0$ and  $n + 0 == n$
            \item Subtraction: $n - n == 0$ and $n - 0 == n$
            \item Multiplication: $n \cdot 0 == 0$ and $n \cdot
                1 == n$
            \item Division: $\frac{n}{n} == 1$ and $\frac{n}{1} == n$
        \end{itemize}

    \item \textbf{Equivalency Cases:} Sometimes, these operators
        become each other. Since a and b must be integers, we don't
        worry about Multiplication becoming Division or Division
        becoming Multiplication. However, Subtracting a negative value
        is just adding the positive, while adding a negative value is
        just subtracting the positive. We can add automated testing
        for these cases.

    \item \textbf{Boundary tests:} There are some boundary tests that
    will need to be implemented:

        \begin{itemize}
            \item Addition: You can't add anything to the maximum
                integer, so we expect an error here. Also, attempting
                to add a negative value to the minimum integer is also
                an error.
            \item Subtraction: You can't subtract a negative value
                from the maximum integer, so we expect an error here.
                Also, attempting to subtract any value from the
                minimum integer is also an error.
            \item Multiplication: The absolute value of the product of
                two integers cannot exceed the integer boundaries
            \item Division: You can't divide by 0.
        \end{itemize}

\end{itemize}

Once we implement all of the specific properties mentioned in a test
case, add some standard unit tests to verify that specific values are
equal to what we expect, and verify we are happy with the testing
suite - We can move on to the code.

Coding these are all relatively simple; each operator is only one
line, after all! However, I wanted to make it explicitly clear that
our code can error from performing specific operations, as those
errors exist in our test cases. I am not going to do any if statements
or anything like that, I am simply going to allow the errors through
to the test case and denote the function can throw the expected error.

\subsection{Post-Coding Modifications}

As feared, Java does allow Overflow and Underflow. Due to this, a decision needs to be made: Either explicitly account for it in all 4 methods (including the given source code), or refactor the test cases to allow the flow edge cases. The former requires more work, but the ladder removes a good chunk of our boundary tests, so both are a costly choice. In the end, I refer to the minimalism point brought up in the discussion about requirements: Imposing a harder to test condition on the code is likely a worse option than removing the test cases. If the overflowed is not desired, we can account for it after acquiring more information.

\end{document}
