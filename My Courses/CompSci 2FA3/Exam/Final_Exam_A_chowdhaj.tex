\documentclass[12pt,fleqn]{article}

\setlength {\topmargin} {-.15in}
\setlength {\textheight} {8.6in}

\usepackage{graphicx}
\usepackage{amsmath}
\usepackage{amssymb}
\usepackage{fancyhdr}
\usepackage{ifthen}
\usepackage{hyperref}
\usepackage{tikz}
\usetikzlibrary{automata,positioning,arrows}
\usepackage{diagbox}
\usepackage{stackrel}

\newcommand{\bc}{\begin{center}}
\newcommand{\ec}{\end{center}}
\newcommand{\bsp}{\begin{sloppypar}}
\newcommand{\esp}{\end{sloppypar}}
\newcommand{\der}[2]{\stackrel[#2]{#1}{\longrightarrow}}
\newcommand{\ForsomeApp}{\exists\,}
\newcommand{\mdot}{\mathrel.}
\newcommand{\sB}{\mbox{$\cal B$}}
\newcommand{\sC}{\mbox{$\cal C$}}
\newcommand{\sD}{\mbox{$\cal D$}}
\newcommand{\sF}{\mbox{$\cal F$}}
\newcommand{\sM}{\mbox{$\cal M$}}
\newcommand{\sP}{\mbox{$\cal P$}}
\newcommand{\sV}{\mbox{$\cal V$}}
\newcommand{\set}[1]{{\{ #1 \}}}

\AtEndDocument{~\label{END}}

\renewcommand{\headrule}{}
\pagestyle{fancyplain}
\rhead[]{}
\lhead[]{}
\fancyfoot[R]{Page~\thepage~of~\pageref{END}}
\fancyfoot[C]{\ifthenelse{\value{page}=\pageref{END}}{THE~END.}{}}

\newcounter{qctr}     % main question counter
\setcounter{qctr}{0}
\newcommand{\question}[1] {  % print a counter and increment it as well
\refstepcounter{qctr}%       % the next label command will ref. this counter
\noindent%
\textbf{%
\rule{0cm}{1cm}%                add a little vertical space
Question \arabic{qctr} [#1]}~~% display counter and a message
}% end question

\begin{document}

\begin{center} 
{\large \bf COMPSCI/SFWRENG 2FA3 Final Exam}\\[1ex]
\end{center}

\medskip

\noindent
DAY CLASS \hfill \textbf{Dr.~W.~M.~Farmer}\\
VERSION: \hfill \textbf{A}\\
DURATION OF EXAMINATION: \hfill \textbf{1.25 hours} \\
MCMASTER UNIVERSITY FINAL EXAMINATION \hfill \textbf{April 22, 2020}

\bigskip

\noindent
\textbf{Academic integrity statement}

\medskip

By submitting this work, I certify that the work represents solely my
own independent efforts. I confirm that I am expected to exhibit
honesty and use ethical behaviour in all aspects of the learning
process.  I confirm that it is my responsibility to understand what
constitutes academic dishonesty under the Academic Integrity Policy
available at

\begin{enumerate}

  \item[] \bsp \href{https://secretariat.mcmaster.ca/app/uploads/Academic-Integrity-Policy-1-1.pdf}{\url{https://secretariat.mcmaster.ca/app/uploads/Academic-Integrity-Policy-1-1.pdf}} \esp

\end{enumerate}

\noindent
\textbf{General instructions}

\medskip

Make a copy of the file \texttt{Final\_Exam\_A.tex} attached to the
Final Exam A folder and name it
\[\texttt{Final\_Exam\_A\_\emph{YourMacID}.tex}\] where \texttt{\emph{YourMacID}}
is your MacID.


The exam consists of 20 short-answer questions.  Answer each of the 20
questions in your \texttt{Final\_Exam\_A\_\emph{YourMacID}.tex}.  Your
answers should be one or at most two sentences long.  Each question is
worth 1 mark.  An answer can receive 0, 0.5, or 1 mark.

\bsp
After you have finished your
\texttt{Final\_Exam\_A\_\emph{YourMacID}.tex} file, use
\texttt{pdflatex} to produce a
\texttt{Final\_Exam\_A\_\emph{YourMacID}.pdf} file.  Submit both your
\texttt{.tex} and \texttt{.pdf} files to the Final Exam A folder.
\esp

You have 75 minutes, starting at 12:30 PM on April 22, to complete the
exam and submit your files.  If your files are submitted to the Final
Exam A folder after 1:45 PM, April~22, 2020, a penalty will be imposed
(see below).

\bigskip

\noindent
\textbf{Special instructions}

\begin{enumerate}

  \item The use of notes, textbooks, and electronic devices is
    permitted.

  \item \textbf{You are \emph{not} allowed to communicate with
    \emph{any} person about the exam questions.}  This means, in
    particular, you are not allowed to obtain help from any person or
    provide help to other students.  Violations of this requirement
    will be processed as a breach of the McMaster University Academic
    Integrity Policy.

  \item Read each question carefully.

  \item Try to allocate your time sensibly and divide it appropriately
    between the questions.

  \item Only the first two sentences of an answer will be read; any
    additional sentences will be ignored.

\end{enumerate}

\noindent
\textbf{Penalties}

\begin{enumerate}

  \item \textbf{If either your .tex file or .pdf file is missing or
    unusable, you will receive a 20\% penalty (i.e., 6 points will be
    subtracted from your final exam mark).}

  \item \textbf{If your files are submitted late, you will receive a
    2\% penalty times the number of minutes the files are late (i.e.,
    6 points will be subtracted from your final exam mark if your
    files are 10 minutes late).}

\end{enumerate}

%%%%%%%%%%%%%%%%%%%%%%%%%%%%%%%%%%%%%%%%%%%%%%%%%%%%%%%%%%%%%%%%%%%%%%

\noindent
\begin{minipage}{\textwidth} 
\question{1 mark} \textbf{Q1}

\bigskip

What is the advantage of a traditional proof over a formal proof?

\bigskip

\textbf{Answer:} 

\bigskip

\emph{The MAIN advantage of traditional proofs over formal proofs is Communication.
In addition, it is good for organization, discovery, and beauty.}

\end{minipage}

%%%%%%%%%%%%%%%%%%%%%%%%%%%%%%%%%%%%%%%%%%%%%%%%%%%%%%%%%%%%%%%%%%%%%%

\noindent
\begin{minipage}{\textwidth} 
\question{1 mark} \textbf{Q2}

\bigskip

What is the advantage of a formal proof over a traditional proof?

\bigskip

\textbf{Answer:} 

\bigskip

\emph{The MAIN advantage of a formal proof over a traditional proof is 
certification because there is a very high assurance that the theorem is correct.
Formal proofs are also good for organization and discovery.}

\end{minipage}

%%%%%%%%%%%%%%%%%%%%%%%%%%%%%%%%%%%%%%%%%%%%%%%%%%%%%%%%%%%%%%%%%%%%%%

\noindent
\begin{minipage}{\textwidth} 
\question{1 mark} \textbf{Q4}

\bigskip

How does one usually prove an existential statement $\ForsomeApp x \in S
\mdot A$?

\bigskip

\textbf{Answer:} 

\bigskip

\emph{For an existential statement, you need to come up with an example for
when 'x' holds in the property 'A'.}

\end{minipage}

%%%%%%%%%%%%%%%%%%%%%%%%%%%%%%%%%%%%%%%%%%%%%%%%%%%%%%%%%%%%%%%%%%%%%%

\noindent
\begin{minipage}{\textwidth} 
\question{1 mark} \textbf{Q9}

\bigskip

What does it mean that an inductive set $S$ has ``no confusion''?


\bigskip

\textbf{Answer:} 

\bigskip

\emph{If $S$ has "no confusion", then each constructor you make with its elements
are unique. In other words, there are no duplicates constructors (or constructors
that are the same). }

\end{minipage}

%%%%%%%%%%%%%%%%%%%%%%%%%%%%%%%%%%%%%%%%%%%%%%%%%%%%%%%%%%%%%%%%%%%%%%

\noindent
\begin{minipage}{\textwidth} 
\question{1 mark} \textbf{Q11}

\bigskip

The mathematical structure $(\mathbb{N},<)$, where $<$ is the usual
order on $\mathbb{N}$, is a well-order.  What is the ordinal induction
principle for $(\mathbb{N},<)$?

\bigskip

\textbf{Answer:} 

\bigskip

\emph{$\forall x \in \mathbb{N} . ((\forall y \in \mathbb{N} . y < x \rightarrow
P(y)) \rightarrow P(x)) \rightarrow \forall x \in \mathbb{N} . P(x)$}

\end{minipage}

%%%%%%%%%%%%%%%%%%%%%%%%%%%%%%%%%%%%%%%%%%%%%%%%%%%%%%%%%%%%%%%%%%%%%%

\noindent
\begin{minipage}{\textwidth} 
\question{1 mark} 

\bigskip

Give an example of a well-order $(U,<)$ such that $U$ is infinite and
the members of $U$ are not natural numbers.

\bigskip

\textbf{Answer:} 

\bigskip

\emph{The Integer (set of numbers) are a good example of this.}

\end{minipage}

%%%%%%%%%%%%%%%%%%%%%%%%%%%%%%%%%%%%%%%%%%%%%%%%%%%%%%%%%%%%%%%%%%%%%%

\noindent
\begin{minipage}{\textwidth} 
\question{1 mark}

\bigskip

Let $\Sigma = ($\sB$,\sC,\sF,\sP,\tau)$ be a signature of
MSFOL where $\sB = \set{\alpha,\beta}$, $\sC = \set{a,b}$, $\sF =
\set{f,g}$, $\sP = \set{p,q}$, and $\tau$ is defined by the following
table: \bc
\begin{tabular}{|c|c|}
\hline
$s$ & $\tau(s)$\\
\hline\hline
$a$ & $\alpha$\\
$b$ & $\beta$\\
$f$ & $(\alpha \rightarrow \beta)$\\
$g$ & $(\beta \rightarrow \alpha)$\\
$p$ & $(\alpha \rightarrow \mathbb{B})$\\
$q$ & $((\alpha \times \beta) \rightarrow \mathbb{B})$\\
\hline
\end{tabular}
\ec

Give an example of a \textbf{closed $\Sigma$-formula that contains the
  universal quantifier but no boolean operators}.  Assume $\sV =
\set{x,y,z,x',y',z',x'',y'',z'',\ldots}$.

\bigskip

\textbf{Answer:} 

\bigskip

\emph{???}

\end{minipage}

%%%%%%%%%%%%%%%%%%%%%%%%%%%%%%%%%%%%%%%%%%%%%%%%%%%%%%%%%%%%%%%%%%%%%%

\noindent
\begin{minipage}{\textwidth} 
\question{1 mark}

\bigskip

Let $\Sigma = (\sB,\sC,\sF,\sP,\tau)$ be a signature of MSFOL; $\sM =
(\sD,I)$ be a $\Sigma$-structure where $\sD = \set{D_\alpha \mid
  \alpha \in \sB}$; and $\phi$ be a variable assignment into $\sM$.
What can you say about the value of $V^{\cal M}_{\phi}(t)$ \textbf{if
  $t$ is a $\Sigma$-term of type $\alpha \in \sB$}?

\bigskip

\textbf{Answer:} 

\bigskip

\emph{The value is satisfiable}

\end{minipage}

%%%%%%%%%%%%%%%%%%%%%%%%%%%%%%%%%%%%%%%%%%%%%%%%%%%%%%%%%%%%%%%%%%%%%%

\noindent
\begin{minipage}{\textwidth} 
\question{1 mark}

\bigskip

What kind of automaton does the following transition table define?
(You don't need to determine the language that the automaton accepts.)

\begin{center}
\begin{tabular}{r|l|ll|}
\cline{2-4}
& {\diagbox{$Q$}{$\Sigma$}} & $0$ & $1$\\
\cline{2-4}
start $\rightarrow$ & $p$ & $\set{q,s}$ & $\set{q}$\\
final $\rightarrow$ & $q$ & $\set{r}$   & $\set{q,r}$\\
                    & $r$ & $\set{s}$   & $\set{p}$\\
final $\rightarrow$ & $s$ & $\set{\,}$  & $\set{p}$\\
\cline{2-4}
\end{tabular}
\end{center}

\bigskip

\textbf{Answer:} 

\bigskip

\emph{This is an NFA (but not with epsilon transitions. not.)}

\end{minipage}

%%%%%%%%%%%%%%%%%%%%%%%%%%%%%%%%%%%%%%%%%%%%%%%%%%%%%%%%%%%%%%%%%%%%%%

\noindent
\begin{minipage}{\textwidth} 
\question{1 mark}

\bigskip

What is Thompson's construction used for?

\bigskip

\textbf{Answer:} 

\bigskip

\emph{Thompson's constructor is used to convert a regular expression into an
NFA, with or without epsilon transitions.}

\end{minipage}

%%%%%%%%%%%%%%%%%%%%%%%%%%%%%%%%%%%%%%%%%%%%%%%%%%%%%%%%%%%%%%%%%%%%%%

\noindent
\begin{minipage}{\textwidth} 
\question{1 mark}

\bigskip

Who first showed that there are undecidable decision problems?

\bigskip

\textbf{Answer:} 

\bigskip

\emph{Alonzo Church first showed this.}

\end{minipage}

%%%%%%%%%%%%%%%%%%%%%%%%%%%%%%%%%%%%%%%%%%%%%%%%%%%%%%%%%%%%%%%%%%%%%%

\noindent
\begin{minipage}{\textwidth} 
\question{1 mark}

\bigskip

Who invented regular expressions?

\bigskip

\textbf{Answer:} 

\bigskip

\emph{Stephen Kleene invented regular expressions.}

\end{minipage}

%%%%%%%%%%%%%%%%%%%%%%%%%%%%%%%%%%%%%%%%%%%%%%%%%%%%%%%%%%%%%%%%%%%%%%

\noindent
\begin{minipage}{\textwidth} 
\question{1 mark}

\bigskip

What is an example of a language that is context-free but not regular?

\bigskip

\textbf{Answer:} 

\bigskip

\emph{An example of context-free grammar is L = $\{a^{n}b^{n} | n > 0\}$}

\end{minipage}

%%%%%%%%%%%%%%%%%%%%%%%%%%%%%%%%%%%%%%%%%%%%%%%%%%%%%%%%%%%%%%%%%%%%%%

\noindent
\begin{minipage}{\textwidth} 
\question{1 mark}

\bigskip

What is an example of a language that is linear but not regular?

\bigskip

\textbf{Answer:} 

\bigskip

\emph{A language that is linear but not regular is: L = $\{a^{n}b^{n}c^{n} | n > 0\}$}

\end{minipage}

%%%%%%%%%%%%%%%%%%%%%%%%%%%%%%%%%%%%%%%%%%%%%%%%%%%%%%%%%%%%%%%%%%%%%%

\noindent
\begin{minipage}{\textwidth} 
\question{1 mark}

\bigskip

What normal form is a context-free grammar in if all its productions
have the form $A \rightarrow aB_1 \cdots B_k$ where $k \ge 0$.

\bigskip

\textbf{Answer:} 

\bigskip

\emph{It is in greibach normal form}

\end{minipage}

%%%%%%%%%%%%%%%%%%%%%%%%%%%%%%%%%%%%%%%%%%%%%%%%%%%%%%%%%%%%%%%%%%%%%%

\noindent
\begin{minipage}{\textwidth} 
\question{1 mark}

\bigskip

Let $M$ be a nondeterministic push-down automaton (NPDA)
$(Q,\Sigma,\Gamma,\delta,s,\bot,F)$.  A configuration of $M$ is a
member of $Q \times \Sigma^* \times \Gamma^*$.  What do the three
components of a configuration designate?

\bigskip

\textbf{Answer:} 

\bigskip

\emph{These three components make up the transition table, which shows you
how the automaton outputs based on each input. For example: If you give it a
value of '0', then the automaton will output 'x' or something else.}

\end{minipage}

%%%%%%%%%%%%%%%%%%%%%%%%%%%%%%%%%%%%%%%%%%%%%%%%%%%%%%%%%%%%%%%%%%%%%%

\noindent
\begin{minipage}{\textwidth} 
\question{1 mark}

\bigskip

Which kind of languages do nondeterministic push-down automata (NPDAs)
accept?

\bigskip

\textbf{Answer:} 

\bigskip

\emph{They accept context-free grammars}

\end{minipage}

%%%%%%%%%%%%%%%%%%%%%%%%%%%%%%%%%%%%%%%%%%%%%%%%%%%%%%%%%%%%%%%%%%%%%%

\noindent
\begin{minipage}{\textwidth} 
\question{1 mark}

\bigskip

What is the tape used for in a Turing machine?

\bigskip

\textbf{Answer:} 

\bigskip

\emph{The tape in a Turing machine is basically memory and stores the program. 
You read from it and
you can write to it, in either direction, and it is semi-infinite. 
Once you reach the END of the tape, the program is over, and the TM halts.}

\end{minipage}

%%%%%%%%%%%%%%%%%%%%%%%%%%%%%%%%%%%%%%%%%%%%%%%%%%%%%%%%%%%%%%%%%%%%%%

\noindent
\begin{minipage}{\textwidth} 
\question{1 mark}

\bigskip

What kind of language is $A$ if the decision problem ``$x \in A$'' is
decidable?

\bigskip

\textbf{Answer:} 

\bigskip

\emph{The language is recursive because there is some total Turing Machine that
accepts the language, and thus the TM will always halt since it is total.}

\end{minipage}

%%%%%%%%%%%%%%%%%%%%%%%%%%%%%%%%%%%%%%%%%%%%%%%%%%%%%%%%%%%%%%%%%%%%%%

\noindent
\begin{minipage}{\textwidth} 
\question{1 mark}

\bigskip

What kind of language is $A$ if the decision problem ``$x \in A$'' is
undecidable?

\bigskip

\textbf{Answer:} 

\bigskip

\emph{The language CAN be recursively enumerable because the program is 
undecidable, and the TM may not halt.}

\end{minipage}

\end{document}

\iffalse

%%%%%%%%%%%%%%%%%%%%%%%%%%%%%%%%%%%%%%%%%%%%%%%%%%%%%%%%%%%%%%%%%%%%%%

\noindent
\begin{minipage}{\textwidth} 
\question{1 mark} \textbf{Q??}

\bigskip

\emph{A question will be written here.}

\bigskip

\textbf{Answer:} 

\bigskip

\emph{Write your answer here.}

\end{minipage}

\fi

