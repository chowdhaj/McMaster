\documentclass[11pt,fleqn]{article}

\usepackage{amsmath}
\usepackage{amssymb}
\usepackage{url}
\usepackage{listings}
\usepackage{color}

\lstset{language=python,basicstyle=\ttfamily,breaklines=true,showspaces=false,showstringspaces=false,breakatwhitespace=true,texcl=true,escapeinside={\%*}{*)}}

\setlength {\topmargin} {-.15in}
\setlength {\textheight} {8.6in}

\renewcommand{\labelenumi}{\theenumi.}
\renewcommand{\labelenumii}{\theenumii.}
\renewcommand{\labelenumiii}{\theenumiii.}
\newcommand{\be}{\begin{enumerate}}
\newcommand{\ee}{\end{enumerate}}
\newcommand{\bi}{\begin{itemize}}
\newcommand{\ei}{\end{itemize}}
\newcommand{\bc}{\begin{center}}
\newcommand{\ec}{\end{center}}
\newcommand{\bsp}{\begin{sloppypar}}
\newcommand{\esp}{\end{sloppypar}}
\newcommand{\sglsp}{\ }
\newcommand{\dblsp}{\ \ }
\newcommand{\mname}[1]{\mbox{\sf #1}}
\newenvironment{proof}{\par\noindent{\bf Proof\sglsp}}{\hfill$\Box$}
\newcommand{\pnote}[1]{{\langle \text{#1} \rangle}} 

\begin{document}

%\thispagestyle{empty}

\bc

  {\large \textbf{COMPSCI/SFWRENG 2FA3}}\\[2mm]
  {\large \textbf{Discrete Mathematics with Applications II}}\\[2mm]
  {\large \textbf{Winter 2020}}\\[8mm]
  {\huge \textbf{Week 02 Exercises with Solutions}}\\[6mm]
  {\large \textbf{Dr.~William M. Farmer}}\\[2mm]
  {\large \textbf{McMaster University}}\\[6mm]
  {\large Revised: January 16, 2020}

\ec

\medskip

\subsection*{Background Definitions}

\begin{enumerate}

  \item The notation $\sum^{n}_{i=m}f(i)$ is defined by: 
    \[\sum^{n}_{i=m}f(i) =
      \left\{\begin{array}{ll}
               0                          & \textrm{if } m > n\\
               f(n) + \sum^{n-1}_{i=m}f(i) & \textrm{if } m \le n
             \end{array}
      \right.\] 

  \item The Fibonacci sequence $\mname{fib} : \mathbb{N} \rightarrow
    \mathbb{N}$ is defined by:
    \[\mname{fib}(n) = 
      \left\{\begin{array}{ll}
               0 & \textrm{if } n = 0 \\
               1 & \textrm{if } n = 1 \\
               \mname{fib}(n-1) + \mname{fib}(n-2) & \textrm{if } n \ge 2
             \end{array}
      \right.\]

   \item Let $a,b \in \mathbb{Z}$.  $a$ \emph{divides} $b$, written $a
     \mid b$, if $b = ac$ for some $c \in \mathbb{Z}$.

\end{enumerate}

\subsection*{Exercises}

\be

  \item Prove the following statements:

  
  \be

    \item The sum of two odd integers is an even integer.

\medskip

\textbf{Solution:}

\medskip

\begin{proof}
Let $m$ and $n$ be arbitrary odd integers.  Since $m$ and $n$ are odd,
there are integers $i$ and $j$ such that $m = 2i + 1$ and $n = 2j +1$.
Then \[m + n = (2i + 1) + (2j + 1) = 2(i + j + 1),\] which show $m +
n$ is even.
\end{proof}

\medskip

    \item If $x$ is an even integer, then $x^2$ is also even.

\medskip

\textbf{Solution:}

\medskip

\begin{proof}
Let $x$ be an arbitrary even integer.  Since $x$ is even, there is some
integer $i$ such that $x = 2i$.  Then \[x^2 = (2i)^2 = 4i^2 =
2(2i^2),\] which shows $x^2$ is even.
\end{proof}

\medskip

    \item Let $a,b,c,d \in \mathbb{Z}$.  If $a \mid b$ and $c \mid d$,
      then $ac \mid bd$. 

\medskip 

\textbf{Solution:}

\medskip

\begin{proof}
Let $a,b,c,d \in \mathbb{Z}$ and assume $a \mid b$ and $c \mid d$.  We
must show $ac \mid bd$.  Since $a \mid b$ and $c \mid d$, there are
integers $i$ and $j$ such that $b = ai$ and $d = cj$.  Then \[bd =
aicj = (ac)(ij),\] which shows $ac \mid bd$.
\end{proof}

\medskip
 
    \item The square root of 2 is an irrational number.

\medskip

\textbf{Solution:}

\medskip

\begin{proof}
Assume the statement is false, i.e., that $\sqrt{2}$ is a rational
number.  This implies that there are integers $m$ and $n$ with no
common divisor other than $\pm 1$ and with $n \not= 0$ such that 
$\sqrt{2}=\frac{m}{n}$.  Then \[2 = (\sqrt{2})^2 = \Big(\frac{m}{n}\Big)^2 =
\frac{m^2}{n^2}.\] Hence $2n^2 = m^2$ since $n \not= 0$.  This implies
$2 \mid m^2$ which implies $2 \mid m$.  Hence there is some integer
$k$ such that $m = 2k$ and so $2n^2 = m^2 = (2k)^2 = 2(2k^2)$.  This
implies $2 \mid n^2$ which implies $2 \mid n$.  Therefore $2 \mid m$
and $2 \mid n$ which contradicts our assumption that $m$ and $n$ have
no common divisor other than $\pm 1$.
\end{proof}

\medskip

  \ee

  \item Prove the following statements by weak induction: 

  \be

    \item $\sum^{n}_{i=0}2i = n(n+1)$ for all $n \in \mathbb{N}$.

\medskip

\textbf{Solution:}

\medskip

\begin{proof}
Let $P(n) \equiv \sum^{n}_{i=0}2*i = n*(n+1)$.  We will prove
$P(n)$ for all $n \in \mathbb{N}$ by weak induction.

\medskip

\emph{Base case}: $n = 0$. Prove $P(0)$.
\begin{align*}
  &\phantom{{}=} \sum^{0}_{i=0}2*i & \pnote{LHS of $P(0)$}\\
  &= 2*0     & \pnote{definition of $\textstyle{\sum}^{n}_{i=m}\,f(i)$ when $m=n$}\\
  &= 0*(0+1) & \pnote{arithmetic; RHS of $P(0)$}
\end{align*}
This shows that $P(0)$ holds.

\medskip

\emph{Induction step}: $n \ge 0$.  Assume $P(n)$. Prove $P(n + 1)$.
\begin{align*}
  &\phantom{{}=} \sum^{n+1}_{i=0}2*i & \pnote{LHS of $P(n + 1)$}\\
  &= 2*(n+1) + \sum^{n}_{i=0}2*i & \pnote{definition of $\textstyle{\sum}^{n}_{i=m}\,f(i)$}\\
  &= 2*(n+1) + n*(n+1) & \pnote{induction hypothesis: $P(n)$}\\
  &= n^2 + 3*n + 2     & \pnote{put in standard form}\\
  &= (n+1)*(n+2)       & \pnote{factor; RHS of $P(n + 1)$}
\end{align*}
This shows that $P(n + 1)$ holds.

\medskip

Therefore, $P(n)$ holds for all $n \in \mathbb{N}$ by weak induction.
\end{proof}

\medskip

    \item $\sum^{n}_{i=1}(2i - 1) = n^2$  for all $n \in \mathbb{N}$.

\medskip

\textbf{Solution:}

\medskip

Similar to 1a.

\medskip

    \item $\sum^{n}_{i=0}i^2 = \frac{n(n+1)(2n + 1)}{6}$ for all
      $n \in \mathbb{N}$.

\medskip

\textbf{Solution:}

\medskip

\begin{proof}
Let $P(n) \equiv \sum^{n}_{i=0}i^2 = \frac{n*(n+1)*(2*n + 1)}{6}$.  We will prove $P(n)$
for all $n \in \mathbb{N}$ by weak induction.

\medskip

\emph{Base case}: $n = 0$.  Prove $P(0)$.
\begin{align*}
  &\phantom{{}=} \sum^{0}_{i=0}i^2 & \pnote{LHS of $P(0)$}\\
  &= 0^2 & \pnote{definition of $\textstyle{\sum}_{i=m}^{n}f(i)$}\\
  &= \frac{0*(0+1)*(2*0 + 1)}{6} & \pnote{arithmetic; RHS of $P(0)$}
\end{align*}
This shows that $P(0)$ holds.

\medskip

\emph{Induction step}: $n \ge 0$.  Assume $P(n)$. Prove $P(n + 1)$.
\begin{align*}
  &\phantom{{}=} \sum^{n+1}_{i=0}i^2\\
    & \pnote{LHS of $P(n + 1)$}\\
  &= (n + 1)^2 + \sum^{n}_{i=0}i^2\\
    &\pnote{definition of $\sum_{i=m}^{n}f(i)$}\\
  &= (n + 1)^2 + \frac{n*(n+1)*(2*n + 1)}{6}\\
    &\pnote{induction hypothesis: $P(n)$}\\
  &= \frac{6*(n + 1)^2 + n*(n+1)*(2*n + 1)}{6}\\
    &\pnote{addition of fractions}\\
  &= \frac{(n + 1)*(6*(n +1) + n*(2*n +1))}{6}\\
    &\pnote{factor out $n + 1$}\\
  &= \frac{(n + 1)*(2*n^2 + 7*n + 6)}{6}\\
    &\pnote{multiply out second factor and collect like terms}\\
  &= \frac{(n + 1)*(n + 2)*(2*n + 3)}{6}\\
    &\pnote{factor second factor}\\
  &= \frac{(n + 1)*(n + 2)*(2*(n + 1) +1)}{6}\\
    &\pnote{arithmetic; RHS of $P(n + 1)$}
\end{align*}
This shows that $P(n + 1)$ holds.

\medskip

Therefore, $P(n)$ holds for all $n \in \mathbb{N}$ by weak induction.
\end{proof}

\medskip

    \item $\sum^{n-1}_{i=0}2^i = 2^n - 1$ for all $n \in \mathbb{N}$.

\medskip

\textbf{Solution:}

\medskip

See the lecture slides for the 1 Mathematical Proof topic.

\medskip

    \item $\sum^{n}_{i=0} \mname{fib}(i) = \mname{fib}(n+2) - 1$ for
      $n \in \mathbb{N}$.

\medskip

\textbf{Solution:}

\medskip

\begin{proof}
Let $P(n) \equiv \sum^{n}_{i=0} \mname{fib}(i) = \mname{fib}(n+2) -
1$.  We will prove $P(n)$ for all $n \in \mathbb{N}$ by weak
induction.

\medskip

\emph{Base case}: $n = 0$. Prove $P(0)$.
\begin{align*}
  &\phantom{{}=} \sum^{0}_{i=0}\mname{fib}(i) & \pnote{LHS of $P(0)$}\\
  &= \mname{fib}(0) & \pnote{def.\ of $\textstyle{\sum}^{n}_{i=m}\,f(i)$ when $m=n$}\\
  &= \mname{fib}(0) + \mname{fib}(1) - 1 & \pnote{arithmetic and definition of \mname{fib}}\\
  &= \mname{fib}(0 + 2) - 1 & \pnote{definition of \mname{fib}; RHS of $P(0)$}
\end{align*}
This shows that $P(0)$ holds.

\medskip

\emph{Induction step}: $n \ge 0$.  Assume $P(n)$. Prove $P(n + 1)$.
\begin{align*}
  &\phantom{{}=} \sum^{n+1}_{i=0}\mname{fib}(i) & \pnote{LHS of $P(n + 1)$}\\
  &= \mname{fib}(n + 1) + \sum^{n}_{i=0}\mname{fib}(i)
    & \pnote{def.\  of $\textstyle{\sum}^{n}_{i=m}\,f(i)$}\\
  &= \mname{fib}(n + 1) + \mname{fib}(n + 2) - 1
    & \pnote{ind.\ hypo.: $P(n)$}\\
  &= \mname{fib}((n + 1) + 2) - 1
    & \pnote{def.\ of \mname{fib}; RHS of $P(n + 1)$}
\end{align*}
This shows that $P(n + 1)$ holds.

\medskip

Therefore, $P(n)$ holds for all $n \in \mathbb{N}$ by weak induction.
\end{proof}

\medskip

    \item $\sum^{n}_{i=0}(\mname{fib}(i))^2 = \mname{fib}(n) *
      \mname{fib}(n+1)$ for all $n \in \mathbb{N}$.

\medskip

\textbf{Solution:}

\medskip

\begin{proof}
Let $P(n) \equiv \sum^{n}_{i=0}(\mname{fib}(i))^2 = \mname{fib}(n) *
\mname{fib}(n+1)$.  We will prove $P(n)$ for all $n \in \mathbb{N}$
by weak induction.
   	
\medskip
   	
\emph{Base case}: $n = 0$. Prove $P(0)$.
\begin{align*}
  &\phantom{{}=} \sum^{0}_{i=0}(\mname{fib}(i))^2 & \pnote{LHS of $P(0)$}\\
  &= (\mname{fib}(0))^2 & \pnote{definition of $\textstyle{\sum}_{i=m}^{n}f(i)$ when $m=n$}\\
  &= 0^2 & \pnote{definition of \mname{fib}}\\
  &= 0 * 1 & \pnote{arithmetic}\\
  &= \mname{fib}(0) * \mname{fib}(0 + 1)& \pnote{definition of \mname{fib}; RHS of $P(0)$}
\end{align*}
This shows that $P(0)$ holds.
	
\medskip
   	
\emph{Induction step}: $n \ge 0$.  Assume $P(n)$.  Prove $P(n + 1)$.
\begin{align*}
  &\phantom{{}=} \sum^{n+1}_{i=1}(\mname{fib}(i))^2 
    & \pnote{LHS of $P(n + 1)$}\\
  &= (\mname{fib}(n+1))^2 + \sum^{n}_{i=1}(\mname{fib}(i))^2
    & \pnote{def.\ of $\textstyle{\sum}_{i=m}^{n}f(i)$}\\
  &= (\mname{fib}(n+1))^2 + \mname{fib}(n) * \mname{fib}(n+1)
    & \pnote{ind.\  hypo.: $P(n)$}\\
  &=  \mname{fib}(n+1)*(\mname{fib}(n+1)+\mname{fib}(n))
    & \pnote{factor out $\mname{fib}(n+1)$}\\
  &= \mname{fib}(n+1)*\mname{fib}(n + 2)
    & \pnote{def.\ of $\mname{fib}$}\\
  &= \mname{fib}(n+1)*\mname{fib}((n + 1) + 1)
    & \pnote{RHS of $P(n + 1)$}\\
\end{align*}
This shows that $P(n + 1)$ holds.

\medskip

Therefore, $P(n)$ holds for all $n \in \mathbb{N}$ by weak induction.
\end{proof}

  \ee

  \item Prove the following statements by strong induction:

  \be

    \item If $n \in \mathbb{N}$ with $n \ge 2$, then $n$ is a product
      of prime numbers.

\medskip

\textbf{Solution:}

\medskip

See the lecture slides for the 1 Mathematical Proof topic.

\medskip

    \item $\mname{fib}(n) < 2^n$ for all $n \in \mathbb{N}$.

\medskip

\textbf{Solution:}

\medskip

\begin{proof}
Let $P(n) \equiv \mname{fib}(n) < 2^n$.  We will prove $P(n)$
for all $n \in \mathbb{N}$ by strong induction.

\medskip

\emph{Base case 1}: $n = 0$. Prove $P(0)$.
\begin{align*}
  &\phantom{{}=} \mname{fib}(0) & \pnote{LHS of $P(0)$}\\
  &= 0                          & \pnote{definition of \mname{fib}}\\
  &< 1                          & \pnote{arithmetic}\\
  &= 2^0                        & \pnote{arithmetic; RHS of $P(0)$}
\end{align*}
This shows that $P(0)$ holds.

\medskip

\emph{Base case 2}: $n = 1$.  Prove $P(1)$.
\begin{align*}
  &\phantom{{}=} \mname{fib}(1) & \pnote{LHS of $P(1)$}\\
 &= 1                           & \pnote{definition of \mname{fib}}\\
 &< 2                           & \pnote{arithmetic}\\
 &= 2^1                         & \pnote{arithmetic; RHS of $P(1)$}
\end{align*}
This shows that $P(1)$ holds.

\medskip

\emph{Induction step}: $n \ge 2$. Assume $P(m)$ for all $m <
n$. Prove $P(n)$.
\begin{align*}
  &\phantom{{}=} \mname{fib}(n)              & \pnote{LHS of $P(n)$}\\
  &= \mname{fib}(n - 1) + \mname{fib}(n - 2) & \pnote{definition of \mname{fib}}\\
  &< 2^{n-1} + \mname{fib}(n - 2)            & \pnote{induction hypothesis: $P(n-1)$}\\
  &< 2^{n-1} + 2^{n-2}                        & \pnote{induction hypothesis: $P(n-2)$}\\
  &< 2^{n-1} + 2^{n-1}                        & \pnote{arithmetic}\\
  &= 2^n                                     & \pnote{arithmetic; RHS of $P(n)$}
\end{align*}
This shows that $P(n)$ holds.

\medskip

Therefore, $P(n)$ holds for all $n \in \mathbb{N}$ by strong induction.
\end{proof}

\medskip

    \item It takes $n - 1$ divisions to break up a rectangular
      chocolate bar containing $n$ squares into individual squares.

\medskip

\textbf{Solution:}

\medskip

\begin{proof}
For a positive integer $n$, let $P(n)$ hold iff every rectangular
chocolate bar containing $n$ squares needs $n - 1$ divisions to be
broken into individual squares.  We will prove $P(n)$ for all $n
\in \mathbb{N}$ with $n \ge 1$ by strong induction.

\medskip

\emph{Base case}: $n = 1$. Prove $P(1)$.  A chocolate bar containing
1 square is already broken into individual squares, and so 0 divisions
are needed to break it up.  This shows that $P(1)$ holds.

\medskip

\emph{Induction step}: $n \ge 2$. Assume $P(m)$ for all $m < n$. Prove
$P(n)$.  Suppose we have a $a \times b$ chocolate bar containing $n =
ab$ squares.  W.l.o.g., we may assume that the first division of the
chocolate bar breaks it into $(a - c) \times b$ and $c \times b$
chocolate bars.  By the induction hypothesis, each of these chocolate
bars can be broken up into individual squares with $(a - c)b - 1$ and
$cb - 1$ divisions, respectively.  Then the number of division needed
to break up the original chocolate bar is \[1 + (a - c)b -1 + cb - 1 =
1 + ab - 1 - 1 = n - 1.\] This shows that $P(n)$ holds.

\medskip

Therefore, $P(n)$ holds for all $n \in \mathbb{N}$ with $n \ge 1$ by
strong induction.
\end{proof}

\medskip

  \ee

  \item Let $t_n,s_n,o_n$ be the $n$th triangle, square, and oblong
    numbers, respectively, where $n \in \mathbb{N}$.

  \be

    \item Define $t_n,s_n,o_n$ by recursion.

    \item Prove by induction that every triangle number is exactly
      half of an oblong number.

    \item Prove by induction that the sum of every two consecutive
      triangle numbers is a square number.

  \ee

\textbf{Solution:}

  \be

    \item $t_n = 
    \left\{\begin{array}{ll}
             0          & \textrm{if } n = 0 \\
             t_{n-1} + n & \textrm{if } n > 0
           \end{array}
    \right.$

    $s_n = 
    \left\{\begin{array}{ll}
             0                 & \textrm{if } n = 0 \\
             s_{n-1} + 2*n - 1  & \textrm{if } n > 0
           \end{array}
    \right.$

    $o_n = 
    \left\{\begin{array}{ll}
             0             & \textrm{if } n = 0 \\
             o_{n-1} + 2*n & \textrm{if } n > 0
           \end{array}
    \right.$

    \item \textbf{Theorem} $t_n = o_n/2$ for all $n \in \mathbb{N}$.

\medskip

\begin{proof}
Let $P(n) \equiv t_n = o_n/2$.  We will prove $P(n)$ for all $n \in
\mathbb{N}$ by weak induction.

\medskip

\emph{Base case}: $n = 0$. Prove $P(0)$.
\begin{align*}
  &\phantom{{}=} t_0 & \pnote{LHS of $P(0)$}\\
  &= 0               & \pnote{definition of $t_n$}\\
  &= 0/2             & \pnote{arithmetic}\\
  &= o_0/2           & \pnote{definition of $o_n$; RHS of $P(0)$}
\end{align*}
This shows that $P(0)$ holds.

\medskip

\emph{Induction step}: $n \ge 0$. Assume $P(n)$. Prove $P(n + 1)$.
\begin{align*}
  &\phantom{{}=} t_{n+1} & \pnote{LHS of $P(n + 1)$}\\
  &= t_n + n + 1         & \pnote{definition of $t_n$}\\
  &= o_n/2 + n + 1       & \pnote{induction hypothesis: $P(n)$}\\
  &= (o_n + 2*(n + 1))/2 & \pnote{arithmetic}\\
  &= o_{n+1}/2           & \pnote{def.\ of $o_n$; RHS of $P(n + 1)$}
\end{align*}
This shows that $P(n + 1)$ holds.

\medskip

Therefore, $P(n)$ holds for all $n \in \mathbb{N}$ by weak induction.
\end{proof}

    \item \textbf{Theorem} $t_n + t_{n+1} = s_{n+1}$ for all $n \in
      \mathbb{N}$.

\medskip

\begin{proof}
Let $P(n) \equiv t_n + t_{n+1} = s_{n+1}$.  We will prove $P(n)$ for
all $n \in \mathbb{N}$ by weak induction.

\medskip

\emph{Base case}: $n = 0$. Prove $P(0)$.
\begin{align*}
  &\phantom{{}=} t_0 + t_1 & \pnote{LHS of $P(0)$}\\
  &= 0 + 1                 & \pnote{definition of $t_n$}\\
  &= 1                     & \pnote{arithmetic}\\
  &= s_1                   & \pnote{definition of $s_n$; LHS of $P(0)$}
\end{align*}
This shows that $P(0)$ holds.

\medskip

\emph{Induction step}: $n \ge 0$. Assume $P(n)$. Prove $P(n + 1)$.
\begin{align*}
  &\phantom{{}=} t_{n+1} + t_{n+2}  & \pnote{LHS of $P(n + 1)$}\\
  &= t_n + n + 1 + t_{n+1} + n + 2  & \pnote{definition of $t_n$}\\
  &= s_{n+1} + n + 1 + n + 2        & \pnote{ind.\ hypo.: $P(n)$}\\
  &= s_{n+1} + 2*(n + 2) - 1        & \pnote{arithmetic}\\
  &= s_{n+2}                        & \pnote{def.\ of $s_n$; RHS of $P(n + 1)$}
\end{align*}
This shows that $P(n + 1)$ holds.

\medskip

Therefore, $P(n)$ holds for all $n \in \mathbb{N}$ by weak induction.
\end{proof}

  \ee

\ee
\end{document}


