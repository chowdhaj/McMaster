\documentclass[11pt,fleqn]{article}

\usepackage{amsmath}
\usepackage{amssymb}
\usepackage{url}
\usepackage{listings}
\usepackage{color}
\usepackage{tabularx}

\lstset{language=python,basicstyle=\ttfamily,breaklines=true,showspaces=false,showstringspaces=false,breakatwhitespace=true,texcl=true,escapeinside={\%*}{*)}}

\setlength {\topmargin} {-.15in}
\setlength {\textheight} {8.6in}

\renewcommand{\labelenumi}{\theenumi.}
\renewcommand{\labelenumii}{\theenumii.}
\renewcommand{\labelenumiii}{\theenumiii.}
\newcommand{\be}{\begin{enumerate}}
\newcommand{\ee}{\end{enumerate}}
\newcommand{\bi}{\begin{itemize}}
\newcommand{\ei}{\end{itemize}}
\newcommand{\bc}{\begin{center}}
\newcommand{\ec}{\end{center}}
\newcommand{\bsp}{\begin{sloppypar}}
\newcommand{\esp}{\end{sloppypar}}
\newcommand{\mname}[1]{\mbox{\sf #1}}
\newcommand{\sB}{\mbox{$\cal B$}}
\newcommand{\sC}{\mbox{$\cal C$}}
\newcommand{\sF}{\mbox{$\cal F$}}
\newcommand{\sM}{\mbox{$\cal M$}}
\newcommand{\sP}{\mbox{$\cal P$}}
\newcommand{\sV}{\mbox{$\cal V$}}
\newcommand{\set}[1]{{\{ #1 \}}}
\newcommand{\Neg}{\neg} 
\ifdefined \And 
\renewcommand{\And}{\wedge}
\else
\newcommand{\And}{\wedge}
\fi
\newcommand{\Or}{\vee}
\newcommand{\Implies}{\Rightarrow}
\newcommand{\Iff}{\LeftRightarrow}
\newcommand{\Forall}{\forall}
\newcommand{\ForallApp}{\forall\,}
\newcommand{\Forsome}{\exists}
\newcommand{\ForsomeApp}{\exists\,}
\newcommand{\mdot}{\mathrel.}

\begin{document}

%\thispagestyle{empty}

\bc

  {\large \textbf{COMPSCI/SFWRENG 2FA3}}\\[2mm]
  {\large \textbf{Discrete Mathematics with Applications II}}\\[2mm]
  {\large \textbf{Winter 2020}}\\[8mm]
  {\huge \textbf{Week 06 Exercises with Solutions}}\\[6mm]
  {\large \textbf{Dr.~William M. Farmer}}\\[2mm]
  {\large \textbf{McMaster University}}\\[6mm]
  {\large Revised: February 16, 2020}

\ec

\medskip

\subsection*{Background Definitions}

Consider the following definitions:

\be

  \item $\Sigma_{\rm stack} = (\sB,\sC,\sF,\sP,\tau)$ where:

  \be

    \item $\sB = \set{\mname{Element},\mname{Stack}}$.

    \item $\sC = \set{\mname{error},\mname{bottom}}$.

    \item $\sF = \set{\mname{push},\mname{pop},\mname{top}}$.

    \item $\sP = \emptyset$.

    \item $\tau(\mname{error}) = \mname{Element}$.

    \item $\tau(\mname{bottom}) = \mname{Stack}$.

    \item $\tau(\mname{push}) = \mname{Element} \times \mname{Stack}
      \rightarrow \mname{Stack}$.

    \item $\tau(\mname{pop}) = \mname{Stack} \rightarrow
      \mname{Stack}$.

    \item $\tau(\mname{top}) = \mname{Stack} \rightarrow
      \mname{Element}$.

  \ee

  \item $\Sigma_{\rm grp} =
    (\set{G},\set{e},\set{*,\mname{inv}},\emptyset,\tau)$ where
    $\tau(e) = G$, $\tau(*) = G\times G \rightarrow G$, and
    $\tau(\mname{inv}) = G \rightarrow G$.

  \item Let $\Gamma_{\rm grp}$ be the following set of
    $\Sigma$-sentences:

  \be

    \item[Assoc] $\ForallApp x,y,z : G \mdot (x * y) * z = x *
      (y * z)$.

    \item[IdLeft] $\ForallApp x : G \mdot e * x = x$.

    \item[IdRight] $\ForallApp x : G \mdot x * e = x$.

    \item[InvLeft] $\ForallApp x : G \mdot \mname{inv}(x) * x = e$.

    \item[InvRight] $\ForallApp x : G \mdot x * \mname{inv}(x) = e$.

  \ee

  \item A \emph{partition} of a set $S$ is a nonempty set $U$ of
    subsets of $S$ such that, for all $x \in S$, $x$ is a member of
    exactly one member of $U$.  Hence (1) the members of $U$ are
    disjoint and (2) their union equals $S$.

  \item A \emph{lattice} is a weak partial order $(U,\le)$ such that
    each pair of elements of $U$ has both a least upper bound and a
    greatest lower bound.

  \item Let $M_1 = (D_1,e_1,*_1)$ and $M_2 = (D_2,e_2,*_2)$ be
    two monoids.  A \emph{monoid homomorphism} from $M_1$ to $M_2$ is
    a function $h : D_1 \rightarrow D_2$ such that:

  \be

    \item $h(x *_1 y) = h(x) *_2 h(y)$ for all $x,y \in D_1$.

    \item $h(e_1) = e_2$.

  \ee

\ee

\subsection*{Exercises}

\be

  \item Construct in MSFOL a theory $T = (\Sigma_{\rm stack},\Gamma_{\sf
    stack})$ of stacks.  $\Gamma_{\sf stack}$ should contain axioms
  that say:

  \be

    \item The top of the bottom stack is the error element.

    \item Let $s$ be a stack obtained by pushing an element $e$ onto a
      stack $s'$.  The top of $s$ is $e$.

    \item Pop of the bottom stack is the bottom stack.

    \item Let $s$ be a stack obtained by pushing an element $e$ onto a
      stack $s'$.  The pop of $s$ is $s'$.

  \ee

  \textbf {Solution:}\\
    $\Sigma_{\rm stack}$ is defined in Background Definitions.\\
    $\Gamma_{\sf stack}$ contains the following:
    \be
      \item $\mname{top(\mname{bottom})} = \mname{error}$
      \item $\ForallApp e : \mname{Element},\,s' : \mname{Stack} \mdot \mname{top}(\mname{push}(e,\,s')) = e$
      \item $\mname{pop(\mname{bottom})} = \mname{bottom}$
      \item $\ForallApp e : \mname{Element},\,s' : \mname{Stack} \mdot \mname{pop}(\mname{push}(e,\,s')) = s'$
    \ee

  \item A \emph{group} is a monoid with an inverse operation.  $T_{\rm
    grp} = (\Sigma_{\rm grp}, \Gamma_{\rm grp})$ is a theory of
    groups.  Show that models of $T_{\rm grp}$ can be directly derived
    from $(\mathbb{Z},0,+)$ and $(\mathbb{Q},1,*)$ but not from
    $(\mathbb{N},0,+)$ and $(\mathbb{Z},1,*)$.

  \textbf{Solution:}
    \begin{quote}
      In each case, we must show/argue that the function provided for $*$ is associative, 
      that the constant provided for $e$ is a left and right identity for that operator, and provide 
      a function for $\mname{inv}$ which is a right and left inverse of the function for $*$.

  \be
    \item[$(\mathbb{Z},0,+)$:] $+$ is associative, $0$ is the identity for $+$, and for $\textsf{inv}$ 
      we may take the unary $-$, since $x + - x = 0$ and $- x + x = 0$.
    \item[$(\mathbb{Q},1,*)$:] Take $\mathbb{Q} \backslash \{ 0 \}$ as $G$, rather than $\mathbb{Q}$, 
      because there is no multiplicative inverse for $0$. Then, $*$ is associative, $1$ is the identity for $*$, 
      and for $\textsf{inv}$ we may take the unary ``recipricol'', since $x * \frac{1}{x} = 1$ and $\frac{1}{x} 
      * x = 1$ (since we have removed $0$).
    \item[$(\mathbb{N},0,+)$:] while $+$ is associative and $0$ is the identity for $+$, we cannot provide an 
      inverse for plus; this can be seen because e.g. $1 + y \neq 0$ for any $y$.
    \item[$(\mathbb{Z},1,*)$:] while $*$ is associative and $1$ is the identity for $*$, we cannot provide an 
      inverse for $*$; this can be seen because e.g. $1 + y \neq 0$ for any $y$..
  \ee
\end{quote}

  \item Let $\Sigma = (\alpha, p : \alpha \rightarrow \mathbb{B}, q :
    \alpha \rightarrow \mathbb{B})$ be a signature of MSFOL.  What
    should $\Gamma$ be so that each model for the theory $T =
    (\Sigma,\Gamma)$ is a set of values partitioned into two
    components defined by $p$ and $q$.

  \textbf{Solution:}

    Let $\Gamma$ contain the following $\Sigma$-sentences:

    \begin{enumerate}

      \item $\Forall x : \alpha \mdot \Neg(p\,x \land q\,x)$.

      \item $\Forall x : \alpha \mdot p\,x \Or q\,x$.

    \end{enumerate}

    Let $\sM = (\set{D_\alpha},I)$ be a model for $T$. Then clearly the
    two sets \[\set{d \in D_\alpha \;|\; V_{\phi[x : \alpha \mapsto
          d]}^{\mbox{M}}p\,(x:\alpha) = T}\] and \[\set{d \in D_\alpha
    \;|\; V_{\phi[x : \alpha \mapsto d]}^{M}q\,(x:\alpha) =
    T}\] are a partition of $D_\alpha$.

  \item Let $\Sigma_{\rm pairs}$ be the signature
    $(\sB,\emptyset,\sF,\emptyset,\tau)$  where:

  \be

    \item $\sB = \set{\alpha,\beta,\gamma}$.

    \item $\sF = \set{\mname{mkPair},\mname{left},\mname{right}}$
      where $\tau(\mname{mkPair}) = \alpha \times \beta \rightarrow
      \gamma$, $\tau(\mname{left}) = \gamma \rightarrow \alpha$, and
      $\tau(\mname{right}) = \gamma \rightarrow \beta$.

  \ee

  What should $\Gamma_{\rm pairs}$ be so that $T_{\rm pairs} =
  (\Sigma_{\rm pairs}, \Gamma_{\rm pairs})$ is a theory of
  mathematical structures that contain (1) sets $A$, $B$, and $C$
  where $C$ is a set of values that have the same structure as ordered
  pairs of members of $A$ and $B$ and (2) functions to construct and
  destruct the pairs in $C$?

\textbf{Solution:}\\
$\Gamma_{\rm pairs}$ should consist of the following sentences:
\be
    \item OnlyPairs   $\Forall z : \gamma \mdot \ForsomeApp x : \alpha , y : \beta \mdot \textsf{mkPair}\ x\ y = z$                                                   
    \item mkPairInj   $\Forall x, x' : \alpha, y, y' : \beta \mdot \textsf{mkPair}\ x\ y = \textsf{mkPair}\ x'\ y' \newline \phantom{x} \Implies (x = x' \And y = y')$ 
     \item IsLeftProj  $\Forall x : \alpha, y : \beta \mdot \textsf{left}\ (\textsf{mkPair}\ x\ y) = x$                                                                
     \item IsRightProj $\Forall x : \alpha, y : \beta \mdot \textsf{right}\ (\textsf{mkPair}\ x\ y) = y$
\ee

  \item Let $\Sigma_{\sf lattice} =
    (\set{U},\emptyset,\emptyset,\set{\le}, \tau)$ where $\tau(\le) =
    U \times U \rightarrow \mathbb{B}$.  Construct in MSFOL a theory
    $T=(\Sigma_{\sf lattice},\Gamma_{\sf lattice})$ of lattices.

  \textit{Solution.}

\qquad $\Sigma_{\sf lattice}$ is defined in Background Definitions.

\qquad $\Gamma_{\sf lattice}$ contain the following:

  \be
	\item $\forall \; x \in U \;.\; x \le x$

	\item $\forall \; x, y \in U \;.\; (x \le y \land y \le x) \Rightarrow  x = y$

	\item $\forall \; x, y, z \in U \;.\; (x \le y \land y \le z) \Rightarrow x \le z$

	\item $\ForallApp x,y \in U \mdot \ForsomeApp z \in U \mdot$\\
          \hspace*{2ex}$x \le z \And y \le z \And (\ForallApp w \in U
          \mdot x \le w \And y \le w \Implies z \le w)$.

	\item $\ForallApp x,y \in U \mdot \ForsomeApp z \in U \mdot$\\
          \hspace*{2ex}$z \le x \And z \le y \And (\ForallApp w \in U
          \mdot w \le x \And w \le y \Implies w \le z)$.

  \ee

  \item Explain why it is not possible to construct a theory of
    well-founded relations in MSFOL.

\textbf{Solution:}
\begin{quote}
  Recall the definition of a well founded relation: a binary relation $R$ on a set $U$ is well founded if every non-empty subset of $U$ has an $R$-minimal element.

  Since it is not possible to quantify over subsets of a sort in MSFOL, it is not possible to construct a theory of well-founded relations in MSFOL.
\end{quote}

  \item Construct in MSFOL a theory of vector spaces.

\textbf{Solution:}
Let $T_{vs} = (\Sigma_{vs},\Gamma_{vs})$ where $\Sigma_{vs}$ contains the following:
\begin{itemize}
  \item $\sB = \set{F, V}$
  \item $\sC = \set{1, \vec{\mathbf{0}}}$
  \item $\sF = \set{*, \cdot, +, +_F, -}$
  \item $\sP = \set{F, V}$
  \item $\tau$ has:
        \begin{itemize}
          \item $\tau(1) = F$
          \item $\tau(\vec{\mathbf{0}}) = V$
          \item $\tau(*) = F \times F \rightarrow F$
          \item $\tau(\cdot) = F \times V \rightarrow V$
          \item $\tau(+) = V \times V \rightarrow V$
          \item $\tau(+_F) = F \times F \rightarrow F$
          \item $\tau(-) = V \rightarrow V$
        \end{itemize}
\end{itemize}

and $\Gamma_{vs}$ contains the following axioms:
\begin{itemize}
  \item $\ForallApp \vec x, \vec y: V \mdot \vec x + \vec y = \vec y + \vec x$
  \item $\ForallApp \vec x, \vec y, \vec z: V \mdot (\vec x + \vec y) + \vec z = \vec x + (\vec y + \vec z)$
  \item $\ForallApp \vec x : V \mdot \vec{\mathbf{0}} + \vec x = \vec x$
  \item $\ForallApp \vec x : V \mdot \vec x + \vec{\mathbf{0}} = \vec x$
  \item $\ForallApp \vec x : V \mdot \vec x + (- \vec x) = \vec{\mathbf{0}}$
  \item $\ForallApp r, s : F,\,\vec x : V \mdot r \mdot (s \cdot \vec x) = (r * s) \cdot \vec x$
  \item $\ForallApp r, s : F,\,\vec x : V \mdot (r +_F s) \cdot \vec x = r \cdot \vec x + s \cdot \vec x$
  \item $\ForallApp r : F,\,\vec x, \vec y : V \mdot r \cdot (\vec x + \vec y) = r \cdot \vec x + r \cdot \vec y$
  \item $\ForallApp \vec x : V \mdot 1 \cdot \vec x = \vec x$
\end{itemize}

  \item Construct in MSFOL a theory $T$ of monoid homomorphisms where
    each model for $T$ contains a monoid homomorphism.

\textbf{Solution:}

Let $\Sigma =
  (\set{M_1,M_2},\set{e_1,e_2},\set{\circ_1,\circ_2,h},\emptyset,\tau)$
where $\tau$ is defined as follows:

\begin{enumerate}

  \item $\tau\,e_1 = M_1$.

  \item $\tau\,e_2 = M_2$.

  \item $\tau\,\circ_1 = M_1 \times M_1 \rightarrow M_1$.

  \item $\tau\,\circ_2 = M_2 \times M_2 \rightarrow M_2$.

  \item $\tau\,h = M_1 \rightarrow M_2$.

\end{enumerate}

Let $T = (\Sigma,\Gamma)$ where $\Gamma$ contains the following
$\Sigma$-sentences:

\begin{enumerate}

  \item $\Forall x,y,z : M_1 \mdot (x \circ_1 y) \circ_1 z = x
          \circ_1 (y \circ_1 z)$.

  \item $\Forall x : M_1 \mdot e \circ_1 x = x$.

  \item $\Forall x : M_1 \mdot x \circ_1 e = x$.

  \item $\Forall x,y,z : M_2 \mdot (x \circ_2 y) \circ_2 z = x
          \circ_2 (y \circ_2 z)$.

  \item $\Forall x : M_2 \mdot e \circ_2 x = x$.

  \item $\Forall x : M_2 \mdot x \circ_2 e = x$.

  \item $\Forall x,y : M_1 \mdot h\,(x \circ_1 y) = (h\,x)
          \circ_2 (h\,y)$.

  \item $h\,e_1 = e_2$.

\end{enumerate}

If \mbox{$\cal H$} = $(\set{D_{M_1},D_{M_2}},I)$ is a model for $T$, then
clearly $I\,h$ is a monoid homomorphism.  If $g$ is a monoid
homomorphism in some mathematical model, then clearly there is a
$\Sigma$-interpretation \mbox{$\cal H$} = $(\set{D_{M_1},D_{M_2}},I)$ such that
$g = I\,h$.  Therefore, $T$ is a theory of monoid homomorphisms.


\ee
\end{document}


