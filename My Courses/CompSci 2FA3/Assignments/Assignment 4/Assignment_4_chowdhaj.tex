\documentclass[11pt,fleqn]{article}

\setlength {\topmargin} {-.15in}
\setlength {\textheight} {8.6in}

\usepackage{amsmath}
\usepackage{amssymb}
\usepackage{amsthm}
\usepackage{color}
\usepackage[dvipsnames]{xcolor}


\renewcommand{\labelenumi}{\theenumi.}
\renewcommand{\labelenumii}{\theenumii.}
\renewcommand{\labelenumiii}{\theenumiii.}
\newcommand{\be}{\begin{enumerate}}
\newcommand{\ee}{\end{enumerate}}
\newcommand{\bi}{\begin{itemize}}
\newcommand{\ei}{\end{itemize}}
\newcommand{\bc}{\begin{center}}
\newcommand{\ec}{\end{center}}
\newcommand{\bsp}{\begin{sloppypar}}
\newcommand{\esp}{\end{sloppypar}}
\newcommand{\mname}[1]{\mbox{\sf #1}}
\newcommand{\sB}{\mbox{$\cal B$}}
\newcommand{\sC}{\mbox{$\cal C$}}
\newcommand{\sF}{\mbox{$\cal F$}}
\newcommand{\sM}{\mbox{$\cal M$}}
\newcommand{\sP}{\mbox{$\cal P$}}
\newcommand{\sV}{\mbox{$\cal V$}}
\newcommand{\set}[1]{{\{ #1 \}}}
\newcommand{\Neg}{\neg} 
\ifdefined \And 
\renewcommand{\And}{\wedge}
\else
\newcommand{\And}{\wedge}
\fi
\newcommand{\Or}{\vee}
\newcommand{\Implies}{\Rightarrow}
\newcommand{\Iff}{\LeftRightarrow}
\newcommand{\Forall}{\forall}
\newcommand{\ForallApp}{\forall\,}
\newcommand{\Forsome}{\exists}
\newcommand{\ForsomeApp}{\exists\,}
\newcommand{\mdot}{\mathrel.}

\begin{document}

\begin{center}

  {\large \textbf{COMPSCI/SFWRENG 2FA3}}\\[2mm]
  {\large \textbf{Discrete Mathematics with Applications II}}\\[2mm]
  {\large \textbf{Winter 2020}}\\[8mm]
  {\huge \textbf{Assignment 4}}\\[6mm]
  {\large \textbf{Dr.~William M. Farmer}}\\[2mm]
  {\large \textbf{McMaster University}}\\[6mm]
  {\large Revised: February 6, 2020}

\end{center}

\medskip

Assignment 4 consists of two problems.  You must write your solutions
to the problems using LaTeX.

Please submit Assignment~4 as two files,
\texttt{Assignment\_4\_\emph{YourMacID}.tex} and
\texttt{Assignment\_4\_\emph{YourMacID}.pdf}, to the Assignment~4
folder on Avenue under Assessments/Assignments.
\texttt{\emph{YourMacID}} must be your personal MacID (written without
capitalization).  The \texttt{Assignment\_4\_\emph{YourMacID}.tex}
file is a copy of the LaTeX source file for this assignment
(\texttt{Assignment\_4.tex} found on Avenue under
Contents/Assignments) with your solution entered after each problem.
The \texttt{Assignment\_4\_\emph{YourMacID}.pdf} is the PDF output
produced by executing

\begin{itemize}

  \item[] \texttt{pdflatex Assignment\_4\_\emph{YourMacID}}

\end{itemize}

This assignment is due \textbf{Sunday, February 16, 2019 before
  midnight.}  You are allow to submit the assignment multiple times,
but only the last submission will be marked.  \textbf{Late submissions
  and files that are not named exactly as specified above will not be
  accepted!}  It is suggested that you submit your preliminary
\texttt{Assignment\_4\_\emph{YourMacID}.tex} and
\texttt{Assignment\_4\_\emph{YourMacID}.pdf} files well before the
deadline so that your mark is not zero if, e.g., your computer fails
at 11:50 PM on February 16.

\textbf{Although you are allowed to receive help from the
  instructional staff and other students, your submission must be your
  own work.  Copying will be treated as academic dishonesty! If any of
  the ideas used in your submission were obtained from other students
  or sources outside of the lectures and tutorials, you must
  acknowledge where or from whom these ideas were obtained.}

%\newpage

\subsection*{Background}

\be  

  \item Let $(S,<)$ be a strict partial order.  $(S,<)$ is
    \emph{dense} if, for all $x,y \in S$ with $x < y$, there is some
    $z \in S$ such that $x < z < y$.

  \item A \emph{queue} is a finite sequence of elements for which
    elements are added (``enqueued'') to the back of the sequence and
    removed (``dequeued'') from the front of the sequence.  An
    \emph{empty queue} is a queue with no members.  A \emph{singleton
      queue} is a queue with a single element that is obtain by
    enqueuing an element to an empty queue.

\ee

\subsection*{Problems}

\be

  \item \textbf{[10 points]} Construct in MSFOL a theory $T$ of strict
    total orders that are dense and have minimum and maximum elements.
    Give two models for $T$.


  \textcolor{blue}{\textbf{Jatin Chowdhary | chowdhaj | Feb. 16th, 2020}}

  \textcolor{blue}{\textbf{Put your solution here.}}

  \textcolor{red}{BEGIN SOLUTION}

    The Strict Total Order $(S,<)$ is dense. So for all \textit{$x,y \in S$} 
    with \textit{$x < y$}, there is some \textit{$z \in S$} such that 
    \textit{$x < z < y$}. 
    
    %This implies that there are at least 3 elements in it. 

    \textit{Let $\mathbb U$ be the universe set that contains all objects including itself. 
    This is the notation I will be using to denote the universe set.}

    Let $\Sigma_{\sf T} = (\sB,\sC,\sF,\sP,\tau)$ where: 
    \be

      \item $\sB = \set{\mname{$\mathbb U$}}$

      \item $\sC = \emptyset$

      \item $\sF = \set{\mname{min}, \mname{max}}$

      $\tau(\mname{min}) = \mname{$\mathbb U$}$.

      $\tau(\mname{max}) = \mname{$\mathbb U$}$.

      \item $\sP = \set{\mname{$<$}}$

      $\tau(\mname{$<$}) = \mname{$\mathbb U \times \mathbb U \rightarrow \mathbb B$}$.

      \medskip

      \ee

      %\textbf{Where $\tau$ maps the set $\sC \cup \sF \cup \sP$ to $\sB$ as follows:}

      %\item $\tau(\mname{min}) = \mname{$\mathbb U$}$.

      %\item $\tau(\mname{max}) = \mname{$\mathbb U$}$.

      %\item $\tau(\mname{$<$}) = \mname{$\mathbb U \times \mathbb U \rightarrow \mathbb B$}$.

      Let $\Sigma_{\sf T}$ be a strict total order and has the following properties:
      
      \textit{The property Dense has already been defined above. The other properties
      are defined in Lecture 3: Predicate Logic, slide 9.}

      \be

        \item Irreflexivity

        \item Transitivity

        \item Asymmetric

        \item Trichotomous

        \item Dense

      \ee

      Two examples of $\Sigma_{\sf T}$ are:

      \begin{itemize}

        \item ($\mathbb N$ : [0, 1000], $<$) 

        \item ($\mathbb Q$ : [69, 420], $<$) 

      \end{itemize}

  \textcolor{red}{END SOLUTION}

  \bigskip

  \newpage

  %%%%%%%%%%%%%%%%%%%%%%%%%%%%%%%%%%%%%%%%%%%%%%%%%%%%%%%%%%%%%%%%%%%%%%%%
  %%%%%%%%%%%%%%%%%%%%%%%%%%%%%%%%%%%%%%%%%%%%%%%%%%%%%%%%%%%%%%%%%%%%%%%%
  %%%%%%%%%%%%%%%%%%%%%%%%%%%%%%%%%%%%%%%%%%%%%%%%%%%%%%%%%%%%%%%%%%%%%%%%

  \item \textbf{[10 points]} Let $\Sigma_{\sf queue} =
    (\sB,\sC,\sF,\sP,\tau)$ where:

  \be

    \item $\sB = \set{\mname{Element},\mname{Queue}}$.

    \item $\sC = \set{\mname{error}, \mname{empty}}$.

    \item $\sF = \set{\mname{front}, \mname{back}, \mname{enqueue},
      \mname{dequeue}}$.

    \item $\sP = \emptyset$.


    \item $\tau(\mname{error}) = \mname{Element}$.

    \item $\tau(\mname{empty}) = \mname{Queue}$.

    \item $\tau(\mname{front}) = \mname{Queue} \rightarrow \mname{Element}$.

    \item $\tau(\mname{back}) = \mname{Queue} \rightarrow \mname{Element}$.

    \item $\tau(\mname{enqueue}) = \mname{Element} \times
      \mname{Queue} \rightarrow \mname{Queue}$.

    \item $\tau(\mname{dequeue}) = \mname{Queue} \rightarrow
      \mname{Queue}$.

  \ee

  Construct in MSFOL a theory $T=(\Sigma_{\sf queue},\Gamma_{\sf
    queue})$ of queues. $\Gamma_{\sf queue}$ should contain axioms
  that say:

  \be

    \item The front of an empty queue is the error element.

    \item The front of a singleton queue is the single element in the queue.

    \item Let $q$ be a queue obtain by enqueuing $e$ to a nonempty
      queue $q'$.  The front of $q$ is the front of $q'$.

    \item The back of an empty queue is the error element.

    \item Let $q$ be a queue obtain by enqueuing $e$ to a queue $q'$.
      The back of $q$ is $e$.

    \item The dequeue of an empty queue is the empty queue.

    \item The dequeue of a singleton queue is the empty queue.

    \item Let $q$ be a queue obtain by enqueuing $e$ to a nonempty
      queue $q'$.  The dequeue of $q$ is the enqueue of $e$ to the
      dequeue of $q'$.

  \ee    

  \textcolor{blue}{\textbf{Jatin Chowdhary | chowdhaj | Feb. 16th, 2020}}

  \textcolor{blue}{\textbf{Put your solution here.}}

  \textcolor{red}{BEGIN SOLUTION}

    $\Sigma_{\sf queue}$ is defined above. 

    $\Gamma_{\sf queue}$ contains the following:

    Let \textit{singleton\_queue} represent a singleton queue, 
    that only contains one element. 
    In other words, \textit{singleton\_queue} is: 

    \textit{enqueue(Element, empty) = Queue}

    \be

      \item front(empty) = error

      \item front(enqueue(Element, empty)) = Element

      \textcolor{Gray}{\textit{I can also write: front(singleton\_queue) = Element}}

      \item $\forall$ e : Element, q' : Queue . front(q') = front(enqueue(e, q'))

      \textcolor{Gray}{\textit{Can be written as: front(q) = front(q')}}

      \item back(empty) = error

      \item $\forall$ e : Element, q' : Queue . back(enqueue(e, q')) = e

      \textcolor{Gray}{\textit{Can be written as: back(q) = e}}

      \item dequeue(empty) = empty

      \item dequeue(enqueue(Element, empty)) = empty

      \item $\forall$ e : Element, q' : Queue . dequeue(enqueue(e, q')) = enqueue(e, dequeue(q'))

      \textcolor{Gray}{\textit{Can be written as: dequeue(q) = enqueue(e, dequeue(q'))}}

    \ee

  \textcolor{red}{END SOLUTION}

\ee

\end{document}