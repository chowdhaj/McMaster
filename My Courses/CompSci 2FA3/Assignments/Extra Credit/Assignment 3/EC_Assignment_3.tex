\documentclass[11pt,fleqn]{article}

\setlength {\topmargin} {-.15in}
\setlength {\textheight} {8.6in}

\usepackage{amsmath}
\usepackage{amssymb}
\usepackage{color}

\renewcommand{\labelenumi}{\theenumi.}
\renewcommand{\labelenumii}{\theenumii.}
\renewcommand{\labelenumiii}{\theenumiii.}
\newcommand{\be}{\begin{enumerate}}
\newcommand{\ee}{\end{enumerate}}
\newcommand{\bi}{\begin{itemize}}
\newcommand{\ei}{\end{itemize}}
\newcommand{\bc}{\begin{center}}
\newcommand{\ec}{\end{center}}
\newcommand{\bsp}{\begin{sloppypar}}
\newcommand{\esp}{\end{sloppypar}}
\newcommand{\mname}[1]{\mbox{\sf #1}}
\newcommand{\sB}{\mbox{$\cal B$}}
\newcommand{\sC}{\mbox{$\cal C$}}
\newcommand{\sE}{\mbox{$\cal E$}}
\newcommand{\sF}{\mbox{$\cal F$}}
\newcommand{\sL}{\mbox{$\cal L$}}
\newcommand{\sM}{\mbox{$\cal M$}}
\newcommand{\sP}{\mbox{$\cal P$}}
\newcommand{\sV}{\mbox{$\cal V$}}
\newcommand{\set}[1]{{\{ #1 \}}}
\newcommand{\Neg}{\neg} 
\ifdefined \And 
\renewcommand{\And}{\wedge}
\else
\newcommand{\And}{\wedge}
\fi
\newcommand{\Or}{\vee}
\newcommand{\Implies}{\Rightarrow}
\newcommand{\Iff}{\Leftrightarrow}
\newcommand{\Forall}{\forall}
\newcommand{\ForallApp}{\forall\,}
\newcommand{\Forsome}{\exists}
\newcommand{\ForsomeApp}{\exists\,}
\newcommand{\mdot}{\mathrel.}

\begin{document}

\begin{center}

  {\large \textbf{COMPSCI/SFWRENG 2FA3}}\\[2mm]
  {\large \textbf{Discrete Mathematics with Applications II}}\\[2mm]
  {\large \textbf{Winter 2020}}\\[8mm]
  {\huge \textbf{Extra Credit Assignment 3}}\\[6mm]
  {\large \textbf{Dr.~William M. Farmer}}\\[2mm]
  {\large \textbf{McMaster University}}\\[6mm]
  {\large Revised: February 14, 2020}

\end{center}

\medskip

Extra Credit Assignment 3 consists of three problems concerning Kleene
algebas.  You must write your solution to the problem using LaTeX.

\bsp
Please submit Extra Credit Assignment~3 as two files,
\texttt{EC\_Assignment\_3\_\emph{YourMacID}.tex} and
\texttt{EC\_Assignment\_3\_\emph{YourMacID}.pdf}, to the Extra Credit
Assignment~3 folder on Avenue under Assessments/Assignments.
\texttt{\emph{YourMacID}} must be your personal MacID (written without
capitalization).  The \texttt{EC\_Assignment\_3\_\emph{YourMacID}.tex}
file is a copy of the LaTeX source file for this assignment
(\texttt{EC\_Assignment\_3.tex} found on Avenue under
Contents/Assignments) with your solution entered after the problem.
The \texttt{EC\_Assignment\_3\_\emph{YourMacID}.pdf} is the PDF output
produced by executing
\esp

\begin{itemize}

  \item[] \texttt{pdflatex EC\_Assignment\_3\_\emph{YourMacID}}

\end{itemize}

This assignment is due \textbf{Sunday, March 8, 2020 before
  midnight.}  You are allow to submit the assignment multiple times,
but only the last submission will be marked.  \textbf{Late submissions
  and files that are not named exactly as specified above will not be
  accepted!}  It is suggested that you submit your preliminary
\texttt{EC\_Assignment\_3\_\emph{YourMacID}.tex} and
\texttt{EC\_Assignment\_3\_\emph{YourMacID}.pdf} files well before the
deadline so that your mark is not zero if, e.g., your computer fails
at 11:50 PM on March 8.

\textbf{Although you are allowed to receive help from the
  instructional staff and other students, your submission must be your
  own work.  Copying will be treated as academic dishonesty! If any of
  the ideas used in your submission were obtained from other students
  or sources outside of the lectures and tutorials, you must
  acknowledge where or from whom these ideas were obtained.}

\newpage

\subsection*{Background}

Let $\Sigma_{\rm ka} =
(\set{K},\set{0,1},\set{+,\cdot,\phantom{}^*},\emptyset,\tau)$ where:

\be

    \item $\tau(0) = \tau(1) = K$.

    \item $\tau(+) = \tau(\cdot) = K \times K \rightarrow K$.

    \item $\tau(\phantom{}^*) = K \rightarrow K$.

\ee
%
We will write $a \cdot b$ as simply $ab$.

Let $T_{\sf ka}=(\Sigma_{\rm ka},\Gamma_{\rm ka})$ be and MSFOL theory
of Kleene algebras where $\Gamma_{\sf ka}$ contain the following
axioms:

\be

  \item $\ForallApp x,y,z : K \mdot x + (y + z) = (x + y) + z$.

  \item $\ForallApp x,y : K \mdot x + y = y + x$.

  \item $\ForallApp x : K \mdot x + 0 = x$.

  \item $\ForallApp x : K \mdot x + x = x$.

  \item $\ForallApp x,y,z : K \mdot x(yz) = (xy)z$.

  \item $\ForallApp x : K \mdot x1 = x$.

  \item $\ForallApp x : K \mdot 1x = x$.

  \item $\ForallApp x : K \mdot x0 = 0$.

  \item $\ForallApp x : K \mdot 0x = 0$.

  \item $\ForallApp x,y,z : K \mdot x(y + z) = xy + xz$.

  \item $\ForallApp x,y,z : K \mdot (x + y)z = xz + yz$.

  \item $\ForallApp x : K \mdot 1 + xx^* = x^*$.

  \item $\ForallApp x : K \mdot 1 + x^*x = x^*$.

  \item $\ForallApp x,y,z : K \mdot y + xz \le z \Implies x^*y \le z$.

  \item $\ForallApp x,y,z : K \mdot y + zx \le z \Implies yx^* \le z$.

\ee

\noindent
Note: $x \le y$ stands for $x + y = y$.

\subsection*{Extra Credit Problem \textbf{[2 bonus points]}}

\be

  \item Prove that the following $\Sigma_{\rm ka}$-structures are
    models of $T_{\sf ka}$:

  \be

    \item $\sM_1 = (\sL, \emptyset, \set{\epsilon}, \cup,
      \mname{concatenation}, \phantom{}^*)$ where $\sL =
      \sP(\Sigma^*)$ is the set of all languages over some alphabet
      $\Sigma$.  Note that technically $\sM_1 = (\set{D_K},I)$ where
      $D_K = \sL$, $I(0) = \emptyset$, $I(1) = \set{\epsilon}$, $I(+)
      = \cup$, $I(\cdot) = \mname{concatenation}$, and
      $I(\phantom{}^*) = \phantom{}^*$.  Notice that $\Sigma_{\rm ka}$
      and $\Sigma$ are very different: $\Sigma_{\rm ka}$ is a
      signature and $\Sigma$ is an alphabet.

    \item $\sM_2 = (\sL_{\rm reg}, \emptyset, \set{\epsilon}, \cup,
      \mname{concatenation}, \phantom{}^*)$ where $\sL_{\rm reg}$ is
      the set of regular languages over some alphabet $\Sigma$.

    \item $\sM_3 = (\sE_{\rm reg}, \emptyset, \epsilon, +, \cdot,
      \phantom{}^*)$ where $\sE_{\rm reg}$ is the set of regular
      expressions over some alphabet $\Sigma$ in which equivalent
      regular expressions are consider equal.

    \item $\sM_4 = (\set{F,T},F,T,\Or,\And,f_T)$ where $F$ and $T$ are
      the standard truth values and $f_T(F) = f_T(T) = T$.

  \ee

  \item Let $(K,0,1,+,\cdot,\phantom{}^*)$ be an arbitrary model of
    $T_{\sf ka}$.  Prove the following:

  \be

    \item $(K,0,+)$ is an idempotent commutative monoid.

    \item $(K,0,1,\cdot)$ is a monoid with an annihilator.

    \item $(K,0,1,+,\cdot)$ is a semiring.

    \item $(K,\le)$ is a weak partial order with least upper bounds
      (what is called a \emph{join-semilattice}).

  \ee

  \item Assuming Axioms 1--13, prove that Axiom 14 and the
    sentence \[\ForallApp x,y: K \mdot xy \le y \Implies x^*y \le
    y\] are logically equivalent.

\ee


\bigskip

\noindent
\textcolor{blue}{\textbf{Put your name, MacID, and date here.}}

\bigskip

\noindent
\textcolor{blue}{\textbf{Put your solution here.}}

\end{document}


