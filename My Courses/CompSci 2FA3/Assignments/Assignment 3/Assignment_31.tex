\documentclass[11pt,fleqn]{article}

\setlength {\topmargin} {-.15in}
\setlength {\textheight} {8.6in}

\usepackage{amsmath}
\usepackage{amssymb}
\usepackage{amsthm}
\usepackage{color}

\renewcommand{\labelenumi}{\theenumi.}
\renewcommand{\labelenumii}{\theenumii.}
\renewcommand{\labelenumiii}{\theenumiii.}
\newcommand{\be}{\begin{enumerate}}
\newcommand{\ee}{\end{enumerate}}
\newcommand{\bi}{\begin{itemize}}
\newcommand{\ei}{\end{itemize}}
\newcommand{\bc}{\begin{center}}
\newcommand{\ec}{\end{center}}
\newcommand{\bsp}{\begin{sloppypar}}
\newcommand{\esp}{\end{sloppypar}}
\newcommand{\mname}[1]{\mbox{\sf #1}}
\newcommand{\pnote}[1]{{\langle \text{#1} \rangle}}
\newcommand{\sB}{\mbox{$\cal B$}}
\newcommand{\sC}{\mbox{$\cal C$}}
\newcommand{\sF}{\mbox{$\cal F$}}
\newcommand{\sP}{\mbox{$\cal P$}}
\newcommand{\subfun}{\sqsubseteq}
\ifdefined \And 
\renewcommand{\And}{\wedge}
\else
\newcommand{\And}{\wedge}
\fi

\begin{document}

\begin{center}

  {\large \textbf{COMPSCI/SFWRENG 2FA3}}\\[2mm]
  {\large \textbf{Discrete Mathematics with Applications II}}\\[2mm]
  {\large \textbf{Winter 2020}}\\[8mm]
  {\huge \textbf{Assignment 3}}\\[6mm]
  {\large \textbf{Dr.~William M. Farmer}}\\[2mm]
  {\large \textbf{McMaster University}}\\[6mm]
  {\large Revised: February 6, 2020}

\end{center}

\medskip

Assignment 3 consists of four problems.  You must write your solutions
to the problems using LaTeX.

Please submit Assignment~3 as two files,
\texttt{Assignment\_3\_\emph{YourMacID}.tex} and
\texttt{Assignment\_3\_\emph{YourMacID}.pdf}, to the Assignment~3
folder on Avenue under Assessments/Assignments.
\texttt{\emph{YourMacID}} must be your personal MacID (written without
capitalization).  The \texttt{Assignment\_3\_\emph{YourMacID}.tex}
file is a copy of the LaTeX source file for this assignment
(\texttt{Assignment\_3.tex} found on Avenue under
Contents/Assignments) with your solution entered after each problem.
The \texttt{Assignment\_3\_\emph{YourMacID}.pdf} is the PDF output
produced by executing

\begin{itemize}

  \item[] \texttt{pdflatex Assignment\_3\_\emph{YourMacID}}

\end{itemize}

This assignment is due \textbf{Sunday, February 9, 2020 before
  midnight.}  You are allow to submit the assignment multiple times,
but only the last submission will be marked.  \textbf{Late submissions
  and files that are not named exactly as specified above will not be
  accepted!}  It is suggested that you submit your preliminary
\texttt{Assignment\_3\_\emph{YourMacID}.tex} and
\texttt{Assignment\_3\_\emph{YourMacID}.pdf} files well before the
deadline so that your mark is not zero if, e.g., your computer fails
at 11:50 PM on February 9.

\textbf{Although you are allowed to receive help from the
  instructional staff and other students, your submission must be your
  own work.  Copying will be treated as academic dishonesty! If any of
  the ideas used in your submission were obtained from other students
  or sources outside of the lectures and tutorials, you must
  acknowledge where or from whom these ideas were obtained.}

\newpage

\subsection*{Background}

Let $\Sigma = (\sB,\sC,\sF,\sP,\tau)$ be a finite signature of MSFOL,
$F_{\Sigma}$ be the set of $\Sigma$-formulas, and $A \in F_{\Sigma}$.
Recall that the members of $F_{\Sigma}$ are certain strings of
symbols.  A \emph{subformula} of $A$ is a $B \in F_{\Sigma}$ such that
$B$ is a substring of $A$.  For example, let $A$ be the formula $((0 =
2) \And (3 \mid 4))$, i.e., $A$ is the string ``$((0 = 2) \And (3 \mid
4))$''.  Then ``$(0 = 2)$'', ``$(3 \mid 4)$'', and ``$((0 = 2) \And (3
\mid 4))$'' are the subformulas of $A$, and ``$(0 = {}$'' and
``$\And$'' are two substrings of $A$ that are not subformulas of $A$.

A function $f : A \rightarrow B$ is \emph{total} if it is defined on
\emph{all} members of $A$.  A function $f : A \rightarrow B$ is a
\emph{partial} if it is be undefined on \emph{some} members of $A$.
For example, the square root function $\sqrt{\cdot} : \mathbb{R}
\rightarrow \mathbb{R}$ is a partial function since $\sqrt{r}$ is
undefined for all $r \in \mathbb{R}$ with $r < 0$. If $f,g : A
\rightarrow B$ are partial or total functions, then $f$ is a
\emph{subfunction} of $g$, written $f \subfun g$, if the domain $D_f$
of $f$ is a subset of the domain of $g$ and, for all $x \in D_f$,
$f(x) = g(x)$.  In other words, $f$ is a subfunction of $g$ if $g(a)$
is defined and $f(a) = g(a)$ whenever $f(a)$ is defined.

\subsection*{Problems}

\be

  \item\textbf{[10 points]} Let $\mname{subformulas} : F_{\Sigma}
    \rightarrow \sP(F_{\Sigma})$ be the function that maps a formula
    $A \in F_{\Sigma}$ to the set of subformulas of $A$.  Define
    $\mname{subformulas}$ by structural recursion using pattern
    matching.

  \textcolor{blue}{\textbf{Jatin Chowdhary | Chowdhaj | February 9th, 2020 }}

  \textcolor{blue}{\textbf{Solution: }}

  Let x\textsubscript{i} $\in$ \textit{T}\textsubscript{$\sum$}.

  Let M, N be formulas that are in our signature. 
  So: M, N $\in$ \textit{F}\textsubscript{$\sum$}.

  \begin{itemize}

  \item\indent{Let \textit{eq} represent Equality.}


  \item\indent{Let \textit{predi} represent Predicate.}


  \item\indent{Let \textit{nega} represent Negation.}


  \item\indent{Let \textit{impli} represent Implication.}


  \item\indent{Let \textit{quants} represent Quantifications. (i.e. $\forall$)}

  \end{itemize}

  \textit{Structural recursion will be defined for all inputs of the above type, and the base case, which is the empty set.}

  \indent {
  subformulas( \{ \} ) = \{ \}
  }

  \medskip

  \indent {
  subformulas( eq(x\textsubscript{e}, x\textsubscript{r}) ) = 
  \{ eq(x\textsubscript{e}, x\textsubscript{r}) \} $\cup$ subforumlas(M)
  }

  \medskip

  \indent {
  subformulas( predi(x\textsubscript{e}, x\textsubscript{r}) ) = 
  \{ predi(x\textsubscript{e}, x\textsubscript{r}) \} 
  
  $\cup$ subforumlas(M)
  }

  \medskip

  \indent {
    subforumlas( nega(M) ) = \{ nega(M) \} $\cup$ subforumlas(M)
  }

  \medskip

  \indent {
  subformulas( impli(M, N) ) = \{ impli(M, N) \} $\cup$ subformulas(M)

  $\cup$ subformulas(N)
  }

  \medskip

  \indent{
    subformulas( quants(z, a, M) ) = \{ quants(z, a, M) \} 

    $\cup$ subformulas(M)
  }

  \item\textbf{[10 points]} Suppose $F$ is the set of partial and
    total functions $f : \mathbb{N} \rightarrow \mathbb{N}$.

  \be

    \item Show that $(F,\subfun)$ is a weak partial order but not a
      weak total order.

      \textcolor{red}{\textit{$(F,\subfun)$ is a weak partial order because it is reflexive, anti-symmetric, and transitive.}}

      \textcolor{red}{\textit{$(F,\subfun)$ is not a weak total order because it 
      is a weak partial order, and the relation is only one sided. If something is
      a partial order, then it can't be a total order. But if it is a total order, 
      than it is also a partial order. And $(F,\subfun)$ is not a total order
      because there are elements in F that are not directly comparable. There is
      more than one way to compare the elements in F, thus making F a partial order.
      Every element in F is not related with one another.}}

    \item Describe the set of minimal elements of $(F,\subfun)$.

      \textcolor{red}{\textit{F does not have any minimal elements
      because it is a weak partial order with an infinite number of
      functions. The minimal elements are all $\subfun$. And because
      of that, there is no minimal element. }}

    \item Describe the set of maximal elements of $(F,\subfun)$.

      \textcolor{red}{\textit{$(F,\subfun)$ does not have any maximal
      elements in it, because there is no maximum or maximal element/number 
      in $\mathbb{N}$. Therefore, $(F,\subfun)$ does have any any max element.}}


    \item Does $(F,\subfun)$ have a minimum element?  If so, what is
      it?

      \textcolor{red}{\textit{No, there is no minimum element, because it 
      is a weak partial order.}}


    \item Does $(F,\subfun)$ have a maximum element?  If so, what is
      it?

      \textcolor{red}{\textit{No, there is no maximum element because there 
      is no number in $\mathbb{N}$ that is greater than every number. In 
      other words, $\mathbb{N}$ is Noetherian and has no maximum element. 
      There is NO upperbound.}}

  \ee

  \textcolor{blue}{\textbf{Jatin Chowdhary | Chowdhaj | February 9th, 2020}}

  \textcolor{blue}{\textbf{Solution is above in italics and red color.}}

\ee

\end{document}


