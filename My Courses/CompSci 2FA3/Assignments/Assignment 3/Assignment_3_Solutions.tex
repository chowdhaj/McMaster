\documentclass[11pt,fleqn]{article}

\setlength {\topmargin} {-.15in}
\setlength {\textheight} {8.6in}

\usepackage{amsmath}
\usepackage{amssymb}
\usepackage{amsthm}
\usepackage{color}

\renewcommand{\labelenumi}{\theenumi.}
\renewcommand{\labelenumii}{\theenumii.}
\renewcommand{\labelenumiii}{\theenumiii.}
\newcommand{\be}{\begin{enumerate}}
\newcommand{\ee}{\end{enumerate}}
\newcommand{\bi}{\begin{itemize}}
\newcommand{\ei}{\end{itemize}}
\newcommand{\bc}{\begin{center}}
\newcommand{\ec}{\end{center}}
\newcommand{\bsp}{\begin{sloppypar}}
\newcommand{\esp}{\end{sloppypar}}
\newcommand{\mname}[1]{\mbox{\sf #1}}
\newcommand{\pnote}[1]{{\langle \text{#1} \rangle}}
\newcommand{\sB}{\mbox{$\cal B$}}
\newcommand{\sC}{\mbox{$\cal C$}}
\newcommand{\sF}{\mbox{$\cal F$}}
\newcommand{\sP}{\mbox{$\cal P$}}
\newcommand{\subfun}{\sqsubseteq}
\newcommand{\set}[1]{{\{ #1 \}}}
\newcommand{\Neg}{\neg} 
\ifdefined \And 
\renewcommand{\And}{\wedge}
\else
\newcommand{\And}{\wedge}
\fi
\ifdefined \And 
\renewcommand{\And}{\wedge}
\else
\newcommand{\And}{\wedge}
\fi
\newcommand{\Or}{\vee}
\newcommand{\ImpliesAlt}{\Rightarrow}
\newcommand{\Forall}{\forall}
\newcommand{\ForallApp}{\forall\,}
\newcommand{\mdot}{\mathrel.}

\begin{document}

\begin{center}

  {\large \textbf{COMPSCI/SFWRENG 2FA3}}\\[2mm]
  {\large \textbf{Discrete Mathematics with Applications II}}\\[2mm]
  {\large \textbf{Winter 2020}}\\[8mm]
  {\huge \textbf{Assignment 3 with Solutions}}\\[6mm]
  {\large \textbf{Dr.~William M. Farmer}}\\[2mm]
  {\large \textbf{McMaster University}}\\[6mm]
  {\large Revised: February 15, 2020}

\end{center}

\medskip

Assignment 3 consists of four problems.  You must write your solutions
to the problems using LaTeX.

Please submit Assignment~3 as two files,
\texttt{Assignment\_3\_\emph{YourMacID}.tex} and
\texttt{Assignment\_3\_\emph{YourMacID}.pdf}, to the Assignment~3
folder on Avenue under Assessments/Assignments.
\texttt{\emph{YourMacID}} must be your personal MacID (written without
capitalization).  The \texttt{Assignment\_3\_\emph{YourMacID}.tex}
file is a copy of the LaTeX source file for this assignment
(\texttt{Assignment\_3.tex} found on Avenue under
Contents/Assignments) with your solution entered after each problem.
The \texttt{Assignment\_3\_\emph{YourMacID}.pdf} is the PDF output
produced by executing

\begin{itemize}

  \item[] \texttt{pdflatex Assignment\_3\_\emph{YourMacID}}

\end{itemize}

This assignment is due \textbf{Sunday, February 9, 2020 before
  midnight.}  You are allow to submit the assignment multiple times,
but only the last submission will be marked.  \textbf{Late submissions
  and files that are not named exactly as specified above will not be
  accepted!}  It is suggested that you submit your preliminary
\texttt{Assignment\_3\_\emph{YourMacID}.tex} and
\texttt{Assignment\_3\_\emph{YourMacID}.pdf} files well before the
deadline so that your mark is not zero if, e.g., your computer fails
at 11:50 PM on February 9.

\textbf{Although you are allowed to receive help from the
  instructional staff and other students, your submission must be your
  own work.  Copying will be treated as academic dishonesty! If any of
  the ideas used in your submission were obtained from other students
  or sources outside of the lectures and tutorials, you must
  acknowledge where or from whom these ideas were obtained.}

\newpage

\subsection*{Background}

Let $\Sigma = (\sB,\sC,\sF,\sP,\tau)$ be a finite signature of MSFOL,
$F_{\Sigma}$ be the set of $\Sigma$-formulas, and $A \in F_{\Sigma}$.
Recall that the members of $F_{\Sigma}$ are certain strings of
symbols.  A \emph{subformula} of $A$ is a $B \in F_{\Sigma}$ such that
$B$ is a substring of $A$.  For example, let $A$ be the formula $((0 =
2) \And (3 \mid 4))$, i.e., $A$ is the string ``$((0 = 2) \And (3 \mid
4))$''.  Then ``$(0 = 2)$'', ``$(3 \mid 4)$'', and ``$((0 = 2) \And (3
\mid 4))$'' are the subformulas of $A$, and ``$(0 = {}$'' and
``$\And$'' are two substrings of $A$ that are not subformulas of $A$.

A function $f : A \rightarrow B$ is \emph{total} if it is defined on
\emph{all} members of $A$.  A function $f : A \rightarrow B$ is a
\emph{partial} if it is be undefined on \emph{some} members of $A$.
For example, the square root function $\sqrt{\cdot} : \mathbb{R}
\rightarrow \mathbb{R}$ is a partial function since $\sqrt{r}$ is
undefined for all $r \in \mathbb{R}$ with $r < 0$. If $f,g : A
\rightarrow B$ are partial or total functions, then $f$ is a
\emph{subfunction} of $g$, written $f \subfun g$, if the domain $D_f$
of $f$ is a subset of the domain of $g$ and, for all $x \in D_f$,
$f(x) = g(x)$.  In other words, $f$ is a subfunction of $g$ if $g(a)$
is defined and $f(a) = g(a)$ whenever $f(a)$ is defined.

\subsection*{Problems}

\be

  \item\textbf{[10 points]} Let $\mname{subformulas} : F_{\Sigma}
    \rightarrow \sP(F_{\Sigma})$ be the function that maps a formula
    $A \in F_{\Sigma}$ to the set of subformulas of $A$.  Define
    $\mname{subformulas}$ by structural recursion using pattern
    matching.

  \textcolor{blue}{\textbf{Put your name, MacID, and date here.}}

\textbf{Solution}:

We define $\mname{subformulas}$ by structural recursion using pattern
matching as follows:

\be

  \item $\mname{subformulas}((s = t)) = \set{(s = t)}$.

  \item $\mname{subformulas}(p(t_1,\ldots,t_n)) = \set{p(t_1,\ldots,t_n)}$.

  \item $\mname{subformulas}(\Neg A) = \set{\Neg A} \cup \mname{subformulas}(A)$.

  \item $\mname{subformulas}((A \ImpliesAlt B)) = {}$ \[\set{(A \ImpliesAlt
    B)} \cup \mname{subformulas}(A) \cup \mname{subformulas}(B).\]

  \item $\mname{subformulas}((\ForallApp x : \alpha \mdot A)) =
    \set{(\ForallApp x : \alpha \mdot A)} \cup \mname{subformulas}(A)$.

\ee

  \item\textbf{[10 points]} Suppose $F$ is the set of partial and
    total functions $f : \mathbb{N} \rightarrow \mathbb{N}$.

  \be

    \item Show that $(F,\subfun)$ is a weak partial order but not a
      weak total order.

    \item Describe the set of minimal elements of $(F,\subfun)$.


    \item Describe the set of maximal elements of $(F,\subfun)$.


    \item Does $(F,\subfun)$ have a minimum element?  If so, what is
      it?


    \item Does $(F,\subfun)$ have a maximum element?  If so, what is
      it?

  \ee

  \textcolor{blue}{\textbf{Put your name, MacID, and date here.}}


\textbf{Solution}:

\be

  \item

\begin{proof} We must show that $(F,\subfun)$ is reflexive, antisymmetric, 
and transitive.

\emph{Reflexivity.}  Obviously, $f \subfun f$ for any $f \in F$, so
$\subfun$ is reflexive.

\emph{Antisymmetry.} Let $f,g \in F$ such that $f \subfun g$ and $g
\subfun f$.  Then (1)~$D_f \subseteq D_g$ and $D_g \subseteq D_f$ and
so $D_f = D_g$ and (2) for all $x \in D_f$, $f(x) = g(x)$.  Therefore,
$f = g$ and so $\subfun$ is antisymmetric.

\emph{Transitivity.} Let $f,g,h \in F$ such that $f \subfun g$ and $g
\subfun h$.  Then (1)~$D_f \subseteq D_g$ and $D_g \subseteq D_h$ and
so $D_f \subseteq D_h$ and (2) for all $x \in D_f$, $f(x) = g(x)$ and
for all $x \in D_g$, $g(x) = h(x)$ and so for all $x \in D_f$, $f(x) =
h(x)$.  Therefore, $f \subfun h$ and so $\subfun$ is transitive.

Therefore, $(F,\subfun)$ is a weak partial order.
\end{proof}

  \item Let $e$ be the empty function.  Then $D_e = \emptyset$.
    Obviously, $e \subfun f$ for all $f \in F$.  Therefore, the set of
    minimal elements is $\set{e}$.

  \item Let $t$ be a total function in $F$.  Then $D_t = \mathbb{N}$.
    Obviously, there is no $f \in F$ except $t$ itself such that $t
    \subfun f$.  Therefore, the set of maximal elements is the
    infinite set of total functions in $F$.

  \item Since the empty function is the only minimal element, it is
    the minimum element in $F$.

  \item Since there is more than one total function in $F$, there is
    no maximum element in $F$.

  \ee

\ee

\end{document}


