\documentclass[11pt,fleqn]{article}

\setlength {\topmargin} {-.15in}
\setlength {\textheight} {8.6in}

\usepackage{amsmath}
\usepackage{amssymb}
\usepackage{color}
\usepackage{tikz}
\usetikzlibrary{automata,positioning,arrows}
\usepackage{diagbox}
\usepackage{stackrel}

\renewcommand{\labelenumi}{\theenumi.}
\renewcommand{\labelenumii}{\theenumii.}
\renewcommand{\labelenumiii}{\theenumiii.}
\newcommand{\be}{\begin{enumerate}}
\newcommand{\ee}{\end{enumerate}}
\newcommand{\bi}{\begin{itemize}}
\newcommand{\ei}{\end{itemize}}
\newcommand{\bc}{\begin{center}}
\newcommand{\ec}{\end{center}}
\newcommand{\bsp}{\begin{sloppypar}}
\newcommand{\esp}{\end{sloppypar}}
\newcommand{\mname}[1]{\mbox{\sf #1}}
\newcommand{\sB}{\mbox{$\cal B$}}
\newcommand{\sC}{\mbox{$\cal C$}}
\newcommand{\sF}{\mbox{$\cal F$}}
\newcommand{\sM}{\mbox{$\cal M$}}
\newcommand{\sP}{\mbox{$\cal P$}}
\newcommand{\sV}{\mbox{$\cal V$}}
\newcommand{\set}[1]{{\{ #1 \}}}
\newcommand{\Neg}{\neg} 
\ifdefined \And 
\renewcommand{\And}{\wedge}
\else
\newcommand{\And}{\wedge}
\fi
\newcommand{\Or}{\vee}
\newcommand{\Implies}{\Rightarrow}
\newcommand{\Iff}{\Leftrightarrow}
\newcommand{\Forall}{\forall}
\newcommand{\ForallApp}{\forall\,}
\newcommand{\Forsome}{\exists}
\newcommand{\ForsomeApp}{\exists\,}
\newcommand{\mdot}{\mathrel.}
\newcommand{\der}[2]{\stackrel[#2]{#1}{\longrightarrow}}

\begin{document}

\begin{center}

  {\large \textbf{COMPSCI/SFWRENG 2FA3}}\\[2mm]
  {\large \textbf{Discrete Mathematics with Applications II}}\\[2mm]
  {\large \textbf{Winter 2020}}\\[8mm]
  {\huge \textbf{Assignment 10}}\\[6mm]
  {\large \textbf{Dr.~William M. Farmer}}\\[2mm]
  {\large \textbf{McMaster University}}\\[6mm]
  {\large Revised: March 24, 2019}

\end{center}

\medskip

Assignment 10 consists of two problems.  You must write your solutions
to the problems using LaTeX.

Please submit Assignment~10 as two files,
\texttt{Assignment\_10\_\emph{YourMacID}.tex} and
\texttt{Assignment\_10\_\emph{YourMacID}.pdf}, to the Assignment~10
folder on Avenue under Assessments/Assignments.
\texttt{\emph{YourMacID}} must be your personal MacID (written without
capitalization).  The \texttt{Assignment\_10\_\emph{YourMacID}.tex}
file is a copy of the LaTeX source file for this assignment
(\texttt{Assignment\_10.tex} found on Avenue under
Contents/Assignments) with your solution entered after each problem.
The \texttt{Assignment\_10\_\emph{YourMacID}.pdf} is the PDF output
produced by executing

\begin{itemize}

  \item[] \texttt{pdflatex Assignment\_10\_\emph{YourMacID}}

\end{itemize}

This assignment is due \textbf{Sunday, April 5, 2020 before
  midnight.}  You are allow to submit the assignment multiple times,
but only the last submission will be marked.  \textbf{Late submissions
  and files that are not named exactly as specified above will not be
  accepted!}  It is suggested that you submit your preliminary
\texttt{Assignment\_10\_\emph{YourMacID}.tex} and
\texttt{Assignment\_10\_\emph{YourMacID}.pdf} files well before the
deadline so that your mark is not zero if, e.g., your computer fails
at 11:50 PM on April 5.

\textbf{Although you are allowed to receive help from the
  instructional staff and other students, your submission must be your
  own work.  Copying will be treated as academic dishonesty! If any of
  the ideas used in your submission were obtained from other students
  or sources outside of the lectures and tutorials, you must
  acknowledge where or from whom these ideas were obtained.}

\newpage

\subsection*{Problems}

\be

  \item \textbf{[10 points]} Construct an NPDA that accepts the language

  \begin{itemize}

    \item[] $\set{a^mb^n \mid m,n \ge 1 \mbox{ with } m \not= n}$.

  \end{itemize}

  \bigskip

  \textcolor{blue}{\textbf{Jatin Chowdhary | chowdhaj | April 5th, 2020}}

  \textcolor{blue}{\textbf{Put the description of your NPDA here.}}

  \bigskip

  $M = (Q, \Sigma, \Gamma, \delta, s, F)$
  where:

  \begin{itemize}

    \item[] $Q = \{S,F\}$
    \item[] $\Sigma = \{a,b\}$
    \item[] $\Gamma = \{a, \perp\}$
    \item[] $\delta = $

      \be
        \item $((S, a, \perp),(S, a \perp))$
        \item $((S, b, \perp),(S, b \perp))$
        \item $((S, a, a),(S, aa))$
        \item $((S, b, a),(f, \epsilon))$
        \item $((F, b, a),(f, \epsilon))$
        \item $((F, b, \perp),(F, \perp))$
        \item $((S, \epsilon, a),(F, \perp))$
        \item $((S, b, \epsilon),(F, \perp))$
      \ee

    \item[] $s = \{S\}$
    \item[] $F = \{F\}$

  \end{itemize}

  \newpage

  \item \textbf{[10 points]} Let $G = (N,\Sigma,P,S)$ be the CFG
    where $N = \set{S,A,B}$, $\Sigma = \set{a,b}$, and $P$ contains the
    following productions:

  \begin{itemize}

    \item[] $S \rightarrow bA \mid aB$.

    \item[] $A \rightarrow bAA \mid aS \mid a$.

    \item[] $B \rightarrow aBB \mid bS \mid b$.

  \end{itemize}

  Construct an NPDA that accepts $L(G)$.
  \bigskip

  \textcolor{blue}{\textbf{Jatin Chowdhary | chowdhaj | April 5th, 2020}}

  \textcolor{blue}{\textbf{Put the description of your NPDA here.}}

  $M = (Q, \Sigma, \Gamma, \delta, s, F)$
  where:

  \begin{itemize}

    \item[] $Q = \{F, L\}$
    \item[] $\Sigma = \{S,A,B,a,b\}$
    \item[] $\Gamma = \{S, a, b\}$
    \item[] $\delta = $

      \be
          \item $((F, a, S),(F, B))$
          \item $((F, a, B),(F, BB))$
          \item $((F, a, A),(F, S))$
          \item $((F, a, A),(L, \epsilon))$
          %%%%%%%%%%%%%%%%%%%%%%%%%%%%%%%%%%%%%%%%%%
          \item $((F, b, S),(F, A))$
          \item $((F, b, B),(F, AA))$
          \item $((F, b, B),(F, S))$
          \item $((F, b, B),(L, \epsilon))$
%        \item $((S, a, \perp),(S, a \perp))$
%        \item $((S, a, a),(S, aa))$
%        \item $((S, a, b),(S, \epsilon))$
%        \item $((S, b, \perp),(S, b \perp))$
%        \item $((S, b, b),(S, bb))$
%        \item $((S, b, a),(S, \epsilon))$
%        \item $((S, \epsilon, \perp),(f, \perp))$
      \ee

    \item[] $s = \{F\}$
    \item[] $F = \{L\}$

  \end{itemize}

\ee

\end{document}