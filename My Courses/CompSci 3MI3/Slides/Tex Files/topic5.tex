\documentclass[11pt]{beamer}
\usetheme{Dresden}
%\usecolortheme{beaver}
\usepackage[utf8]{inputenc}
\usepackage{amsmath}
\usepackage{amsfonts}
\usepackage{amssymb}
\usepackage{hyperref}
\usepackage{graphicx}
\usepackage{listings}
\usepackage{verbatim}
\usepackage{mathpartir}
\usepackage{ebproof}
\usepackage{xcolor}
\usepackage{syntax}
\author{NCC Moore}

\title{Topic 5 - Untyped Lambda Calculus}
\institute{McMaster University} 
\date{Fall 2021} 
\subject{COMPSCI 3MI3 - Principles of Programming Languages} 
\stepcounter{section}




\definecolor{mGreen}{rgb}{0,0.6,0}
\definecolor{mGray}{rgb}{0.5,0.5,0.5}
\definecolor{mPurple}{rgb}{0.58,0,0.05}
\definecolor{mGreen2}{rgb}{0.05,0.65,0.05}
\definecolor{mGray2}{rgb}{0.55,0.55,0.55}
\definecolor{mPurple2}{rgb}{0.63,0.05,0.05}
\definecolor{backgroundColour}{rgb}{0.95,0.95,0.92}
\definecolor{backgroundColour2}{rgb}{0.95,0.92,0.95}

\lstdefinestyle{C}{
    backgroundcolor=\color{backgroundColour},   
    commentstyle=\color{mGreen},
    keywordstyle=\color{blue},
    numberstyle=\tiny\color{mGray},
    stringstyle=\color{mPurple},    
    basicstyle=\footnotesize,
    breakatwhitespace=false,         
    breaklines=true,                 
    captionpos=b,                    
    keepspaces=true,                 
    numbers=left,                    
    numbersep=5pt,                  
    showspaces=false,                
    showstringspaces=false,
    showtabs=false,                  
    tabsize=2,
    language=C
}

\definecolor{burntOrange}{rgb}{1, 0.4, 0.1} 
\usecolortheme[named=burntOrange]{structure} 

\begin{document}

\begin{frame}
\center
COMPSCI 3MI3 - Principles of Programming Languages
\titlepage

Adapted from ``Types and Programming Languages'' by Benjamin C. Pierce 
\end{frame}

\begin{frame}
\tableofcontents
\end{frame}

\section[Intro]{Introduction to Lambda-Calculus}

\begin{frame}[fragile=singleslide]{}
\begin{center}
\includegraphics[scale=0.4]{figures/thinklambda.jpg}
\end{center}
\end{frame}

\begin{frame}[fragile=singleslide]{Computation my Friends! Computation!}
In the 1960s, Peter Landin observed that complex programming languages can be understood by capturing their essential mechanisms as a small core calculus.  
\begin{itemize}
\item The core language used by Landin was \textbf{$\lambda$-Calculus}
\begin{itemize}
\item Developed in the 1920s by Alonzo Church.
\item Reduces \emph{all} computation to function \textbf{definition} and \textbf{application}.
\end{itemize}
\end{itemize}
The strength of $\lambda$-Calculus comes from it's \emph{simplicity} and its capacity for \textbf{formal reasoning}. 
\end{frame}

\begin{frame}[fragile=singleslide]{Languages Based on $\lambda$-Calculus }
\begin{center}
\includegraphics[width=0.3\textwidth]{figures/haskell.png} 
\includegraphics[width=0.3\textwidth]{figures/Ocaml.png} 
\includegraphics[width=0.3\textwidth]{figures/lisp.png} \\
Haskell \hspace{14em} Lisp \\
\vspace{2em}
\end{center}
\end{frame}

\begin{frame}[fragile=singleslide]{}
\begin{center}
Inclusion of lambda expressions in multi-paradigm languages has been a trend recently...
\includegraphics[width=\textwidth]{figures/excel-lambda.png}
\end{center}
\end{frame}

\section[Basics]{The Basics}
\begin{frame}[fragile=singleslide]{The Tactile (As Opposed to Visual) Basics}
Abstraction is the programmer's most reliable weapon.  For a true programmer, the following statement should be instinctively irritating:
\begin{lstlisting}[style=C]
(5*4*3*2*1) + (7*6*5*4*3*2*1) + (3*2*1)
\end{lstlisting}
Your humble professor couldn't even type the above without using copy-paste.  Our instinct tells us to rewrite the above as:
\begin{lstlisting}[style=C]
factorial(n) = if n = 0 then 1 else n * factorial (n-1) 
factorial(5) + factorial(7) + factorial(3)
\end{lstlisting}
\end{frame}

\begin{frame}[fragile=singleslide]{Baby's First $\lambda$}
\begin{lstlisting}[style=C]
factorial(n) = if n = 0 then 1 else n * factorial (n-1) 
\end{lstlisting}
Right now, the left-hand side is doing too much work (that is, any at all).  Let's introduce a new operator, $\lambda$, which does the work of \texttt{(n)} in the above.
{\small
\begin{equation}
\texttt{factorial = } \lambda \texttt{n . if n = 0 then 1 else n * factorial (n-1)}
\end{equation}}
In Haskell, this would be written:
\begin{lstlisting}[style=C, language=Haskell]
factorial = 
  (\ n -> if n = 0 then 1 else (factorial (n-1)))
\end{lstlisting}
\begin{itemize}
\item In equation 1, \textbf{function application} is in the traditional $f(x)$ form, whereas Haskell and $\lambda$-Calculus use a space character for function application.
\end{itemize}
\end{frame}

\begin{frame}[fragile=singleslide]{$\lambda$-Calculus}
In $\lambda$-Calculus, \emph{everything} is either a function definition or a function application of this form.  
\begin{itemize}
\item The arguments accepted by functions are functions
\item The results returned by functions are also functions.  
\end{itemize}
\begin{center}
\includegraphics[scale=0.2]{figures/recursion.jpg}
\end{center}

\end{frame}


\begin{frame}[fragile=singleslide]{$\lambda$-Calculus Syntax}
Untyped $\lambda$-Calculus is comprised of only 3 terms! \\
\dotfill
\begin{grammar}
<t> ::= <x> 
\alt $\lambda$<x> . <t>
\alt <t> <t>
\end{grammar}
\dotfill \\
These terms are:
\begin{itemize}
\item Variable names
\item $\lambda$ Abstraction
\item Function Application.
\end{itemize}
\end{frame}


\section[Typography]{Typographic Details}
\begin{frame}[fragile=singleslide]{}
\begin{center}
\includegraphics[height=0.8\textheight]{figures/layers.jpeg}
\end{center}
\end{frame}

\begin{frame}[fragile=singleslide]{Layers of Syntax}
When working with programming languages, it is useful to be able to re-organize, and even transform our syntax before applying semantics to it.  It is very common to distinguish:
\begin{itemize}
\item \textbf{Concrete Syntax}
\begin{itemize}
\item The syntax the programmer actually encodes the program in.
\end{itemize}
\item from \textbf{Abstract Syntax}
\begin{itemize}
\item A tree structure containing the terms of the program
\end{itemize}
\end{itemize}
It is sometimes useful to specify even more layers than these.  
\begin{itemize}
\item The syntax of a complex language can often be vastly simplified via purely syntactic transformations.
\item Redundant constructs can be simplified to reduce the number of distinct terms (aka \textbf{desugaring})
\begin{itemize}
\item e.g., changing array access to pointer arithmetic in C
\end{itemize}
\end{itemize}
\end{frame}

\begin{frame}[fragile=singleslide]{AST}
Abstract syntax is an excellent way of visualizing a program's structure, especially in resolving operator precedence.  
\begin{itemize}
\item For example, under BEDMAS, the expression $1+2*3$ would be the left diagram, not the right diagram:
\end{itemize}
\begin{center}
\includegraphics[scale=0.3]{figures/ast1.png} \\
BEDMAS trees are evaluated leaf-first, but as we'll see, in $\lambda$ expressions may be evaluated using a number of different strategies.  
\end{center}
\end{frame}


\begin{frame}[fragile=singleslide]{ASTs of $\lambda$-Calculus}
In order to reduce the number of redundant parentheses in our concrete syntax for $\lambda$-Calculus:
\begin{itemize}
\item Function application will be \textbf{left-associative}.  That is, \texttt{s t u} is interpretted as: \\
\end{itemize}

\begin{center}
\includegraphics[scale=0.3]{figures/ast2.png}
\end{center}
\begin{itemize}
\item or, in concrete syntax, \texttt{(s t) u}
\end{itemize}
\end{frame}

\begin{frame}[fragile=singleslide]{Scope of $\lambda$ Operator}
The abstraction operator $\lambda$ is taken to extend to the right as far as possible.  For the following expression:
\begin{itemize}
\item $\lambda$\texttt{x.}$\lambda$\texttt{y.x y x}
\end{itemize}
We would construct an AST:

\begin{center}
\includegraphics[scale=0.4]{figures/ast3.png}
\end{center}

\end{frame}

\begin{frame}[fragile=singleslide]{Free vs Bound Variables}
In predicate calculus, we recognize a distinction between \textbf{free} and \textbf{bound} variables.
\begin{equation}
\exists x\:|\:x \neq y
\end{equation}
In the above:
\begin{itemize}
\item $x$ is \textbf{bound} by the existential quantifier.
\item $y$ is not bound by a quantifier and is therefore \textbf{free}
\end{itemize}
In $lambda$-Calculus, we apply the same concepts and the same terms to the relationship between variables and the abstraction operator $\lambda$.
\begin{equation}
(\lambda x . x\:y)\:x
\end{equation}
\begin{itemize}
\item The first occurance of $x$ is \textbf{bound}.
\item Both $y$ and the second occurrance of $x$ are \textbf{free}.
\end{itemize}
\end{frame}


\section[Semantics]{Operational Semantics}
\begin{frame}[fragile=singleslide]{Operational Semantics}
\begin{center}
\includegraphics[height=0.8\textheight]{figures/atheism.jpg}
\end{center}
\end{frame}

\begin{frame}[fragile=singleslide]{Only One Evaluation Rule!}
Each execution step performs a function application on a term with at least one abstracted variable.  
\begin{itemize}
\item This means both a $\lambda$ abstraction \emph{and} a function application must be present and adjacent for a term to be reducible.  
\end{itemize}
These terms reduce by \textbf{substituting} the abstracted variable with the term applied to the function.  In other words:
\begin{equation}
(\lambda x. t_1)\:t_2 \rightarrow [x \mapsto t_2]\:t_1
\end{equation}
\begin{itemize}
\item A $\lambda$ expression which may be simplified is known as a \textbf{redex}, or \emph{reducible expression}.
\item The above evaluation process is known as \textbf{beta-reduction}.
\end{itemize}

\end{frame}

\begin{frame}[fragile=singleslide]{Using All our Substitutions}
In the previous slide, the symbol $[x \mapsto t_2]\:t_1$ stands for ``the term obtained by the replacement of all free occurances of $x$ in $t_1$ by $t_2$.
\begin{itemize}
\item We will eventually need to define \emph{two} sets of operational semantics, one for rewriting lambda expressions, and another for performing substitutions.  
\end{itemize}
Examples:
\begin{equation}
(\lambda x.x)\: y \rightarrow y
\end{equation}
\begin{equation}
(\lambda x . x\:(\lambda x . x))\:(u\:r) \rightarrow u\:r\:(\lambda x . x)
\end{equation}
Note in this last example that the substitution operation does not pass to the inner $\lambda$ expression.  This is because occurances of $x$ inside this expression are not \textbf{free}, but \textbf{bound} to the containing abstraction.  
\end{frame}


\begin{frame}[fragile=singleslide]{Evaluation Dilemma!}
So far, we have a reasonably rigorous definiton for beta reduction, and our intuitions about substitution derived from our high-school algebra classes.  
\begin{itemize}
\item The goal is be able to create an algorithm which evaluates lambda expressions.  
\item What happens if we have a choice of multiple beta-reductions in a single $\lambda$ expression?  
\item We need an \emph{evaluation strategy}, which we can build into our operational semantics. 
\end{itemize}

\end{frame}


\begin{frame}[fragile=singleslide]{Our Test Expression}
To examine strategies, we will use a running example expression:
\begin{equation}
(\lambda x. x)\:((\lambda x.x)\:(\lambda z. (\lambda x . x)\:z))
\end{equation}
\begin{itemize}
\item $\lambda x. x$ is effectively an \textbf{identity function}, so we write it as $id$.
\end{itemize}
\begin{equation}
id\:(id\:(\lambda z. id\:z))
\end{equation}
The above expression has three redexes:
\begin{equation}
{\color{red} id\:(id\:(\lambda z. id\:z))}
\end{equation}

\begin{equation}
id\:({\color{red} id\:(\lambda z. id\:z)})
\end{equation}
\\
\begin{equation}
id\:(id\:(\lambda z. {\color{red}id\:z}))
\end{equation}

\end{frame}

\begin{frame}[fragile=singleslide]{The Worst Strategy Ever}
Under \textbf{Full Beta-Reduction}, the redexes may be reduced in any order.  
\begin{itemize}
\item Full beta-reduction is not really even a stategy.
\item This algorithm is non-deterministic.
\end{itemize}
\begin{center}
\includegraphics[scale=0.25]{figures/Full-Reduction.jpg}
\end{center}
\end{frame}

\begin{frame}[fragile=singleslide]{Normal Order}
\textbf{Normal order} begins with the leftmost, outermost redex, and proceeds until there are no more redexes to evaluate. This is the way a human would probably choose to it if they weren't thinking about it too hard. \\
\vspace{1em}
\begin{center}
\begin{tabular}{c l}
& ${\color{red} id\:(id\:(\lambda z. id\:z))}$ \\ 
$\rightarrow$ & ${\color{red} id\:(\lambda z. id\:z)}$ \\ 
$\rightarrow$ & $\lambda z. {\color{red} id\:z}$ \\
$\rightarrow$ & $\lambda z. z$ \\
$\nrightarrow$ &  \\
\end{tabular}
\end{center}
Under this strategy (and those to follow), evaluation is a \emph{partial function}, as each term $t$ evaluates to \emph{at most} one term $t'$
\end{frame}

\begin{frame}[fragile=singleslide]{Call By Name}
The \textbf{call by name} strategy is more restrictive than normal order.  You can't evaluate anything that isn't an outer-most term.  \\
\vspace{1em}
\begin{center}
\begin{tabular}{c l}
& ${\color{red} id\:(id\:(\lambda z. id\:z))}$ \\ 
$\rightarrow$ & ${\color{red} id\:(\lambda z. id\:z)}$ \\ 
$\rightarrow$ & $\lambda z. id\:z$ \\
$\nrightarrow$ &  \\
\end{tabular}
\end{center}
In this case, $\lambda z. id\:z$ is considered a \textbf{normal form}.  

\end{frame}

\begin{frame}[fragile=singleslide]{Haskell is Cool!}
An optimized version of call by name strategy, called \textbf{call by need} is used by Haskell to evaluate expressions.  
\begin{itemize}
\item In order to avoid having to re-evaluate the arguments of expressions, Haskell overwrites all occurances of an expression the first time that expression is evaluated.  
\item As a result, they only need to be evaluated \emph{once}.
\item Effectively, this is a reduction relation on syntax \textbf{graphs}, rather than syntax \textbf{trees}.
\end{itemize}

\end{frame}


\begin{frame}[fragile=singleslide]{Call By Value}
Most languages use \textbf{call by value}, where only the outermost redexes are reduced, and a redex is only reduced when the right-hand-side has already been reduced to a value.
\begin{itemize}
\item Here, as elsewhere, a value is a term in normal form.
\end{itemize}
\vspace{1em}
\begin{center}
\begin{tabular}{c l}
& $ id\: {\color{red}(id\:(\lambda z. id\:z))}$ \\ 
$\rightarrow$ & ${\color{red} id\:(\lambda z. id\:z)}$ \\ 
$\rightarrow$ & $\lambda z. id\:z$ \\
$\nrightarrow$ &  \\
\end{tabular}
\end{center}

We will be using this strategy a lot, because it is commonly implemented in programming languages, and easier to enrich with added features.  
\end{frame}

\begin{frame}[fragile=singleslide]{Curry in a Hurry!}
You may have noticed that so far our functions have only taken one argument.  
\begin{itemize}
\item It would be almost trivial to define an extension to our calculus which allows multiple arguments.
\item We don't have to, however, because of \textbf{currying}.
\end{itemize}
Because our functions are \textbf{higher order functions}, that is, they can return a function as their result, we can describe a function taking multiple arguments as a series of functions taking one argument, that pass their result to the next.  
\end{frame}


\begin{frame}[fragile=singleslide]{Coconut Lamb Curry!}
This is how we might pass multiple arguments in a richer language:

\begin{equation}
(\lambda(x,y,z). s) (a,b,c) \rightarrow [x \mapsto a][y \mapsto b][z \mapsto c]s
\end{equation}

In our calculus, the following statement is equivalent. \\
\vspace{1em}
\begin{center}
\begin{tabular}{c l}
& $(\lambda x.\lambda y.\lambda z.\:s)\:a\:b\:c$ \\ 
$\rightarrow$ & $(\lambda y.\lambda z.\:[x \mapsto a]s)\:b\:c$ \\
$\rightarrow$ & $(\lambda z.\:[x \mapsto a][y \mapsto b]s)\:c$ \\
$\rightarrow$ & $[x \mapsto a][y \mapsto b][z \mapsto c]s)$ \\
$\nrightarrow$ & \\
\end{tabular} \\
It should be noted that our untyped $\lambda$-Calculus has not been designed for use by programmers, but for greater simplicity in proving mathematical properties.  
\end{center}

\end{frame}


\section[Booleans]{Church Booleans}
\begin{frame}[fragile=singleslide]{Church Booleans}
\begin{center}
\includegraphics[width=\textwidth]{figures/Currying.png}
\end{center}
\end{frame}

\begin{frame}[fragile=singleslide]{Strong Like Bool}
Now that we have our new mode of computation established, let's start reconstructing the elements of UAE, starting with the Booleans!
\begin{itemize}
\item We can define true and false values as follows.
\begin{itemize}
\item We use \texttt{tru} and \texttt{fls} here to avoid confusion with the \texttt{true} and \texttt{false} of UAE.
\end{itemize}
\end{itemize}
\begin{equation}
\texttt{tru} = \lambda t. \lambda f. t
\end{equation}
\begin{equation}
\texttt{fls} = \lambda t. \lambda f. f
\end{equation}
\end{frame}

\begin{frame}[fragile=singleslide]{Wait, What?}
Our definitions of Boolean values won't make a lot of sense until we show how they're used.  Consider the following $\lambda$ expression, reproducing UAE's \texttt{if then else} term:
\begin{equation}
\texttt{test} = \lambda t_1. \lambda t_2. \lambda t_3.\:t_1\:t_2\:t_3
\end{equation}

\begin{columns}
\begin{column}{0.5\textwidth}
\begin{center}
\underline{With $t_1 = \texttt{tru}$}
\begin{tabular}{c l}
& $(\lambda t_1. \lambda t_2. \lambda t_3.\:t_1\:t_2\:t_3)\:\texttt{tru}\:u\:v$ \\ 
$\rightarrow$ & $(\lambda t_2.\lambda t_3.\:\texttt{tru}\:t_2\:t_3)\:u\:v$ \\
$\rightarrow$ & $(\lambda t_3.\:\texttt{tru}\:u\:t_3)\:v$ \\
$\rightarrow$ & $\texttt{tru}\:u\:v$ \\
$\rightarrow$ & $(\lambda t. \lambda f. t)\:u\:v$ \\
$\rightarrow$ & $(\lambda f. u)\:v$ \\
$\rightarrow$ & $u$ \\
$\nrightarrow$ & \\
\end{tabular}
\end{center}

\end{column}
\begin{column}{0.5\textwidth}
\begin{center}
\underline{With $t_1 = \texttt{fls}$}
\begin{tabular}{c l}
& $(\lambda t_1. \lambda t_2. \lambda t_3.\:t_1\:t_2\:t_3)\:\texttt{fls}\:u\:v$ \\ 
$\rightarrow$ & $(\lambda t_2.\lambda t_3.\:\texttt{fls}\:t_2\:t_3)\:u\:v$ \\
$\rightarrow$ & $(\lambda t_3.\:\texttt{fls}\:u\:t_3)\:v$ \\
$\rightarrow$ & $\texttt{fls}\:u\:v$ \\
$\rightarrow$ & $(\lambda t. \lambda f. f)\:u\:v$ \\
$\rightarrow$ & $(\lambda f. f)\:v$ \\
$\rightarrow$ & $v$ \\
$\nrightarrow$ & \\
\end{tabular}
\end{center}
\end{column}
\end{columns}
\end{frame}

\begin{frame}[fragile=singleslide]{Boolean Operators}
Adding pieces to $\lambda$-Calculus is very different from adding pieces to UAE.   
\begin{itemize}
\item To expand UAE, we needed to add additional terms and evaluation rules. 
\begin{itemize}
\item The more terms and evaluation rules we add, the longer and more complicated our proofs become! 
\end{itemize}
\item By contrast, when we ``added'' Booleans to our $\lambda$-Calculus, \emph{nothing actually had to be added to the language itself!}
\begin{itemize}
\item \texttt{tru} and \texttt{fls} are not terms, but \textbf{labels} for $\lambda$ expressions \emph{that were already valid terms!}
\end{itemize}
\end{itemize}
\end{frame}

\begin{frame}[fragile=singleslide]{Conservative Extension}
Consider two theories, $T_1$ and $T_2$.  We say that $T_2$ is a \textbf{conservative extension} of $T_1$ if:
\begin{itemize}
\item Every theorem of $T_1$ is a theorem of $T_2$
\item Any theorem of $T_2$ in the language of $T_1$ is already a theorem of $T_1$.
\end{itemize}
The relationship between $\lambda$-Calculus, and $\lambda$-Calculus with Booleans fits the above description.
\begin{itemize}
\item We did not have to introduce any additional theorems to describe the Booleans.  
\item Our Boolean extension still has all the rules of $\lambda$-Calculus.
\end{itemize}
The reason this is useful, is that anything proven about $\lambda$-Calculus is \emph{automatically true} of any conservative extension! 
\end{frame}

\begin{frame}[fragile=singleslide]{Boolean And}
Since adding language elements is so easy under $\lambda$-Calculus, let's add a few more! 
\begin{equation}
\texttt{and} = \lambda b. \lambda c.\:b\:c\:fls
\end{equation}

\begin{columns}
\begin{column}{0.5\textwidth}
\begin{center}
\underline{With input \texttt{tru} \texttt{tru}}
\begin{tabular}{c l}
& $(\lambda b. \lambda c.\:b\:c\:\texttt{fls})\:\texttt{tru}\:\texttt{tru}$ \\ 
$\rightarrow$ & $(\lambda c.\:\texttt{tru}\:c\:\texttt{fls})\:\texttt{tru}$ \\
$\rightarrow$ & $\texttt{tru}\:\texttt{tru}\:\texttt{fls}$ \\
$\rightarrow$ & $(\lambda t. \lambda f. t)\:\texttt{tru}\:\texttt{fls}$ \\
$\rightarrow$ & $(\lambda f. \texttt{tru})\:\texttt{fls}$ \\
$\rightarrow$ & $\texttt{tru}$ \\
$\nrightarrow$ & \\
\end{tabular}
\end{center}

\end{column}
\begin{column}{0.5\textwidth}
\begin{center}
\underline{With input \texttt{tru} \texttt{fls}}
\begin{tabular}{c l}
& $(\lambda b. \lambda c.\:b\:c\:\texttt{fls})\:\texttt{tru}\:\texttt{fls}$ \\ 
$\rightarrow$ & $(\lambda c.\:\texttt{tru}\:c\:\texttt{fls})\:\texttt{fls}$ \\
$\rightarrow$ & $\texttt{tru}\:\texttt{fls}\:\texttt{fls}$ \\
$\rightarrow$ & $(\lambda t. \lambda f. t)\:\texttt{fls}\:\texttt{fls}$ \\
$\rightarrow$ & $(\lambda f. \texttt{fls})\:\texttt{fls}$ \\
$\rightarrow$ & $\texttt{fls}$ \\
$\nrightarrow$ & \\
\end{tabular}
\end{center}
\end{column}
\end{columns}
\end{frame}



\begin{frame}[fragile=singleslide]{Completing Our Truth Table}
\begin{columns}
\begin{column}{0.5\textwidth}
\begin{center}
\underline{With input \texttt{fls} \texttt{tru}}
\begin{tabular}{c l}
& $(\lambda b. \lambda c.\:b\:c\:\texttt{fls})\:\texttt{fls}\:\texttt{tru}$ \\ 
$\rightarrow$ & $(\lambda c.\:\texttt{fls}\:c\:\texttt{fls})\:\texttt{tru}$ \\
$\rightarrow$ & $\texttt{fls}\:\texttt{tru}\:\texttt{fls}$ \\
$\rightarrow$ & $(\lambda t. \lambda f. f)\:\texttt{tru}\:\texttt{fls}$ \\
$\rightarrow$ & $(\lambda f. f)\:\texttt{fls}$ \\
$\rightarrow$ & $\texttt{fls}$ \\
$\nrightarrow$ & \\
\end{tabular}
\end{center}

\end{column}
\begin{column}{0.5\textwidth}
\begin{center}
\underline{With input \texttt{fls} \texttt{fls}}
\begin{tabular}{c l}
& $(\lambda b. \lambda c.\:b\:c\:\texttt{fls})\:\texttt{fls}\:\texttt{fls}$ \\ 
$\rightarrow$ & $(\lambda c.\:\texttt{fls}\:c\:\texttt{fls})\:\texttt{fls}$ \\
$\rightarrow$ & $\texttt{fls}\:\texttt{fls}\:\texttt{fls}$ \\
$\rightarrow$ & $(\lambda t. \lambda f. f)\:\texttt{fls}\:\texttt{fls}$ \\
$\rightarrow$ & $(\lambda f. f)\:\texttt{fls}$ \\
$\rightarrow$ & $\texttt{fls}$ \\
$\nrightarrow$ & \\
\end{tabular}
\end{center}
\end{column}
\end{columns}
\end{frame}

\begin{frame}[fragile=singleslide]{Pairs}
Using the selectivity of Church Booleans, we can easily use them to encode \textbf{pairs}.
\begin{equation}
\texttt{pair} = \lambda f. \lambda s. \lambda b.\:b\:f\:s
\end{equation}
\begin{equation}
\texttt{fst} = \lambda p.\:p\:tru 
\end{equation}
\begin{equation}
\texttt{snd} = \lambda p.\:p\:fls 
\end{equation}
\begin{itemize}
\item $b$ is used to select between $f$ and $s$
\item \texttt{fst} and \texttt{snd} merely apply \texttt{tru} and \texttt{fls} respectively.
\item Since \texttt{tru} selects the first argument, it also selects the first term in the pair.
\item Likewise for \texttt{fls}
\end{itemize}
\end{frame}

\section[Numerals]{Church Numerals}
\begin{frame}[fragile=singleslide]{Church Numerals}
\begin{center}
\includegraphics[scale=0.35]{figures/cortex.png}
\end{center}
\end{frame}

\begin{frame}[fragile=singleslide]{Church Numerals}
We can define the natural numbers in $\lambda$-Calculus in a manner still quite similar to Peano arithmetic.  

\begin{align}
c_0 &= \lambda s. \lambda z.\:z \\
c_1 &= \lambda s. \lambda z.\:s\:z \\
c_2 &= \lambda s. \lambda z.\:s\:(s\:z) \\
c_3 &= \lambda s. \lambda z.\:s\:(s\:(s\:z)) \\
\vdots \nonumber
\end{align}
In other words, Church numerals take two arguments, a successor $s$ and a zero term $z$. 
\begin{itemize}
\item $z$ is applied to $s$, and the result is applied to another $s$ and so on, until we reach $n$ applications.
\end{itemize}
\end{frame}

\begin{frame}[fragile=singleslide]{Correspondance with Booleans}

The observant student may have noticed that $c_0$ has the same definition as \texttt{fls}.

\begin{itemize}
\item This is sometimes called a \textbf{pun} in computer science.
\item The same thing occurs in lower level languages, where the interpretation of a sequence of bits is context dependant. 
\item In C, the bit arrangement 0x00000000 corresponds to:
\begin{itemize}
\item Zero (Integer)
\item False (Boolean)
\item "\textbackslash 0\textbackslash 0\textbackslash 0\textbackslash 0" (Character Array)
\end{itemize}
\end{itemize}

\end{frame}

\begin{frame}[fragile=singleslide]{\texttt{Succ}-ess!}
We can define the successor function on Church Numerals as follows:
\begin{equation}
\texttt{succ} = \lambda n. \lambda s. \lambda z.\:s\:(n\:s\:z)
\end{equation}

\begin{center}
\underline{Successor of Two} \\
\begin{tabular}{c l}
& $\texttt{succ } c_2$ \\ 
$\rightarrow$ & $(\lambda n. \lambda s. \lambda z.\:s\:(n\:s\:z))\:c_2$ \\
$\rightarrow$ & $\lambda s. \lambda z.\:s\:(c_2\:s\:z)$ \\
$\rightarrow$ & $\lambda s. \lambda z.\:s\:((\lambda s. \lambda z.\:s\:(s\:z))\:s\:z)$ \\
$\rightarrow$ & $\lambda s. \lambda z.\:s\:((\lambda z.\:s\:(s\:z))\:z)$ \\
$\rightarrow$ & $\lambda s. \lambda z.\:s\:(s\:(s\:z))$ \\
$\rightarrow$ & $c_3$ \\
$\nrightarrow$ & \\
\end{tabular}
\end{center}

\end{frame}

\begin{frame}[fragile=singleslide]{\texttt{Add}-itional Functions!}
Similarily, we can define addition as follows:
\begin{equation}
\texttt{plus} = \lambda m. \lambda n. \lambda s. \lambda z.\:m\:s(n\:s\:z)
\end{equation}

\begin{center}
\underline{Freedom is the freedom to say...} \\
\begin{tabular}{c l}
& $\texttt{plus}\:c_2\:c_2$ \\ 
$\rightarrow$ & $(\lambda m. \lambda n. \lambda s. \lambda z.\:m\:s\:(n\:s\:z)) c_2 c_2$ \\
$\rightarrow$ & $(\lambda n. \lambda s. \lambda z.\:c_2\:s\:(n\:s\:z)) c_2$ \\
$\rightarrow$ & $\lambda s. \lambda z.\:c_2\:s\:(c_2\:s\:z)$ \\
$\rightarrow$ & $\lambda s. \lambda z.\:(\lambda s. \lambda z.\:s\:(s\:z))\:s\:((\lambda s. \lambda z.\:s\:(s\:z))\:s\:z)$ \\
$\rightarrow$ & $\lambda s. \lambda z.\:(\lambda z.\:s\:(s\:z))\:((\lambda s. \lambda z.\:s\:(s\:z))\:s\:z)$ \\
$\rightarrow$ & $\lambda s. \lambda z.\:(s\:(s\:((\lambda s. \lambda z.\:s\:(s\:z))\:s\:z)))$ \\
$\rightarrow$ & $\lambda s. \lambda z.\:(s\:(s\:((\lambda z.\:s\:(s\:z))\:z)))$ \\
$\rightarrow$ & $\lambda s. \lambda z.\:(s\:(s\:(s\:(s\:z))))$ \\
$\rightarrow$ & $c_4$ \\
$\nrightarrow$ & \\
\end{tabular}
\end{center}
\end{frame}

\begin{frame}[fragile=singleslide]{\texttt{Times} Have Changed}
Finally, let's define a multiplication operator.  
\begin{equation}
times = \lambda m. \lambda n.\:m\:(\texttt{plus}\:n)\:c_0 
\end{equation}

\begin{center}
\underline{$3 \times 2 = ?$} \\
\begin{tabular}{c l}
& $\texttt{times}\:c_3\:c_2$ \\ 
$\rightarrow$ & $(\lambda m. \lambda n.\:m\:(\texttt{plus}\:n)\:c_0)\:c_3\:c_2$ \\
$\rightarrow$ & $(\lambda n.\:c_3\:(\texttt{plus}\:n)\:c_0)\:c_2$ \\
$\rightarrow$ & $(\lambda s. \lambda z.\:s\:(s\:(s\:z)))\:(\texttt{plus}\:c_2)\:c_0$ \\
$\rightarrow$ & $(\texttt{plus}\:c_2)\:((\texttt{plus}\:c_2)\:((\texttt{plus}\:c_2)\:c_0))$ \\
\end{tabular}
\end{center}
\end{frame}

\begin{frame}[fragile=singleslide]{Sub-Derivation}
Technically this is cheating, since we don't have a rule for this type of substitution in the semantic, and it violates our evaluation strategy.
\begin{center}
\begin{tabular}{c l}
& $\texttt{plus}\:c_2$ \\ 
$\rightarrow$ & $(\lambda m. \lambda n. \lambda s. \lambda z.\:m\:s\:(n\:s\:z))\:(\lambda s. \lambda z.\:s\:(s\:z))$ \\ 
$\rightarrow$ & $(\lambda n. \lambda s. \lambda z.\:(\lambda s. \lambda z.\:s\:(s\:z))\:s\:(n\:s\:z))$ \\
$\rightarrow$ & $(\lambda n. \lambda s. \lambda z.\:(\lambda z.\:s\:(s\:z))\:(n\:s\:z))$ \\
$\rightarrow$ & $(\lambda n. \lambda s. \lambda z.\:(s\:(s\:(n\:s\:z))))$ \\
\end{tabular}
\end{center}
(It saves a lot of time though)
\end{frame}

\begin{frame}[fragile=singleslide]{}
\begin{center}
\begin{tabular}{c l}
& $(\texttt{plus}\:c_2)\:((\texttt{plus}\:c_2)\:((\texttt{plus}\:c_2)\:c_0))$ \\ 
$\rightsquigarrow$ & $(\lambda n. \lambda s. \lambda z.\:(s\:(s\:(n\:s\:z))))\:((\texttt{plus}\:c_2)\:((\texttt{plus}\:c_2)\:c_0))$ \\ 
$\rightarrow$ & $\lambda s. \lambda z.\:(s\:(s\:(((\texttt{plus}\:c_2)\:((\texttt{plus}\:c_2)\:c_0))\:s\:z)))$ \\ 
$\rightsquigarrow$ & $\lambda s. \lambda z.\:(s\:(s\:(((\lambda n. \lambda s. \lambda z.\:(s\:(s\:(n\:s\:z))))\:((\texttt{plus}\:c_2)\:c_0))\:s\:z)))\:$ \\ 
$\rightarrow$ & $\lambda s. \lambda z.\:(s\:(s\:((\lambda z.\:(s\:(s\:(((\texttt{plus}\:c_2)\:c_0)\:s\:z))))\:z)))\:$ \\ 
$\rightarrow$ & $\lambda s. \lambda z.\:(s\:(s\:(s\:(s\:(((\texttt{plus}\:c_2)\:c_0)\:s\:z)))))$ \\ 
$\rightsquigarrow$ & $\lambda s. \lambda z.\:(s\:(s\:(s\:(s\:(((\lambda n. \lambda s. \lambda z.\:(s\:(s\:(n\:s\:z))))\:c_0)\:s\:z)))))$ \\ 
$\rightarrow$ & $\lambda s. \lambda z.\:(s\:(s\:(s\:(s\:((\lambda s. \lambda z.\:(s\:(s\:(c_0\:s\:z))))\:s\:z)))))$ \\ 
$\rightarrow$ & $\lambda s. \lambda z.\:(s\:(s\:(s\:(s\:((\lambda z.\:(s\:(s\:(c_0\:s\:z))))\:z)))))$ \\ 
$\rightarrow$ & $\lambda s. \lambda z.\:(s\:(s\:(s\:(s\:(s\:(s\:(c_0\:s\:z)))))))$ \\
$\rightarrow$ & $\lambda s. \lambda z.\:(s\:(s\:(s\:(s\:(s\:(s\:((\lambda s. \lambda z.\:z)\:s\:z)))))))$ \\
$\rightarrow$ & $\lambda s. \lambda z.\:(s\:(s\:(s\:(s\:(s\:(s\:((\lambda z.\:z)\:z)))))))$ \\
$\rightarrow$ & $\lambda s. \lambda z.\:(s\:(s\:(s\:(s\:(s\:(s\:z))))))$ \\
$\nrightarrow$ & \\
\end{tabular}
\end{center}

\end{frame}

\begin{frame}[fragile=singleslide]{Last Slide Comic}
\begin{center}
\includegraphics[height=0.8\textheight]{figures/math-cartoon.jpg}
\end{center}
\end{frame}

\end{document}
