\documentclass{exam}
\usepackage[utf8]{inputenc}
\usepackage{listings}
\usepackage{graphicx}
\usepackage{hyperref}
\usepackage{xcolor}
\usepackage{listings}

\let\OldTexttt\texttt

\renewcommand{\texttt}[1]{\OldTexttt{\color{teal}{#1}}}
\definecolor{mGreen}{rgb}{0,0.6,0}
\definecolor{mGray}{rgb}{0.5,0.5,0.5}
\definecolor{mPurple}{rgb}{0.58,0,0.05}
\definecolor{mGreen2}{rgb}{0.05,0.65,0.05}
\definecolor{mGray2}{rgb}{0.55,0.55,0.55}
\definecolor{mPurple2}{rgb}{0.63,0.05,0.05}
\definecolor{backgroundColour}{rgb}{0.95,0.95,0.92}
\definecolor{backgroundColour2}{rgb}{0.95,0.92,0.95}

\lstdefinestyle{C}{
    backgroundcolor=\color{backgroundColour},   
    commentstyle=\color{mGreen},
    keywordstyle=\color{blue},
    numberstyle=\tiny\color{mGray},
    stringstyle=\color{mPurple},    
    basicstyle=\footnotesize,
    breakatwhitespace=false,         
    breaklines=true,                 
    captionpos=b,                    
    keepspaces=true,                 
    numbers=left,                    
    numbersep=5pt,                  
    showspaces=false,                
    showstringspaces=false,
    showtabs=false,                  
    tabsize=2,
    language=C
}

\definecolor{t_comment}{rgb}{0.2,1,0.2}
\definecolor{t_mGray}{rgb}{0.5,0.5,0.5}
\definecolor{t_mPurple}{rgb}{0.58,0,0.05}
\definecolor{t_blue}{rgb}{0.4,0.6,0.8}
\definecolor{t_mGreen2}{rgb}{0.05,0.65,0.05}
\definecolor{t_mGray2}{rgb}{0.75,0.75,0.75}
\definecolor{t_mPurple2}{rgb}{0.63,0.05,0.05}
\definecolor{t_bg}{rgb}{0.15,0.15,0.18}

\lstdefinestyle{terminal}{
    backgroundcolor=\color{t_bg},   
    commentstyle=\color{t_comment},
    keywordstyle=\color{t_blue},
    numberstyle=\tiny\color{t_mGray},
    stringstyle=\color{t_mGray2}, 
    basicstyle=\footnotesize\color{t_mGray2},
    breakatwhitespace=false,         
    breaklines=true,                 
    captionpos=b,                    
    keepspaces=true,                 
    numbers=none,                    
    numbersep=5pt,                  
    showspaces=false,                
    showstringspaces=false,
    showtabs=false,                  
    tabsize=2,
    language=C
}

\graphicspath{{.}}
\begin{document}

\begin{center}
\fbox{\fbox{\parbox{5.5in}{\centering
\Large{COMPSCI 3MI3 : Assignment 2} \\
\large{Fall 2021} \\ 
Nicholas Moore}}}
\end{center}

\begin{questions}
\question[4] \textbf{Question 1 : Hey Jude} \\
Consider the following diagram: \\
\begin{center}
\includegraphics[scale=0.4]{figures/jude.jpg}
\end{center}
Write an EBNF grammar which completely describes the grammar of the song ``Hey Jude'' by the Beatles, as given in the diagram above.  
\question[6] \textbf{Question 2 : Syntax Rapid Fire!} \\
Write EBNF grammars for the following.
\begin{itemize}
\item Identifiers in Python
\item Identifiers in Haskell
\item For loop in C. You don't have to write grammar for statements or expressions.  
\end{itemize}
Note: This question expects you to do a little research to find out what characters are valid in Python and Haskell identifiers.  
\question[10] \textbf{Grammar in More Depth} \\
Define the following list of terms in Python using the inference rule style we discussed in lecture.  You do not need to define any terms other than the ones given.
\begin{itemize}
\item for loop
\item while loop
\item Assignment statement
\item Slicing operation with all start and end point, as well as interval specified.
\item function declaration
\end{itemize}
\question \textbf{Implementation of UAE in Haskell} \\
\begin{parts}
\part[3] \textbf{Defining Our Terms} \\
Using Haskell's Type System, create an implementation of Untyped Arithmetic Expressions, as described in lecture.  You should implement this language as a custom data type.  Please have whatever types you define derive the Eq and Show typeclasses.  \emph{You don't have to implement any semantics or type checking.}  These are topics we'll be covering later in the course, and you'll get plenty of practice in later assignments.  

Note: Transcribing the term names directly from the textbook isn't going to work, so you'll have to modify them slightly to avoid conflicts with some Haskell keywords like \texttt{True} and \texttt{False}.  You are free to choose whatever modifications you like to the term names, just make sure your terms are still understandable as the originals.  
\part[3] \textbf{Writing Untyped Arithmetic Expressions}  
Write Haskell functions which encode the following arithmetic expressions.  These functions should constants of the term type.
\begin{equation}
\texttt{succ}\: (\texttt{succ}\: (\texttt{succ}\: 0)) 
\end{equation}
\begin{equation}
\texttt{succ}\:(\texttt{pred}\: (\texttt{iszero} (\texttt{succ true})))
\end{equation}
\begin{equation}
\texttt{if} \: (\texttt{iszero}\: (\texttt{succ}\: (\texttt{pred} \: 0)))\: \texttt{then}\: (\texttt{pred}\: \texttt{false})\: \texttt{else}\: (\texttt{succ}\: (\texttt{succ}\: (\texttt{succ}\: (\texttt{succ}\: 0))))
\end{equation}

Once you have written these expressions, load your Haskell file into ghci (because who needs the hassle of adding a main function).  If you've done everything correctly, this file should typecheck, and you should even be able to write arithmetic expressions in ghci using your brand-new term datatype.  

\part[4] \textbf{Wait, Isn't that an Error?}
While syntactically correct, the expressions in the previous question contained some semantic errors.  Find all such errors, and explain why the error is an error.  
\end{parts}

% No hints in the tex file? What a shame! I guess you'll just have to use stackoverflow...

\end{questions}
\end{document}