 \documentclass{exam}
\usepackage[utf8]{inputenc}
\usepackage{listings}
\usepackage{graphicx}
\usepackage{hyperref}
\usepackage{xcolor}
\usepackage{listings}
\usepackage{amsmath}

\let\OldTexttt\texttt

\renewcommand{\texttt}[1]{\OldTexttt{\color{teal}{#1}}}
\definecolor{mGreen}{rgb}{0,0.6,0}
\definecolor{mGray}{rgb}{0.5,0.5,0.5}
\definecolor{mPurple}{rgb}{0.58,0,0.05}
\definecolor{mGreen2}{rgb}{0.05,0.65,0.05}
\definecolor{mGray2}{rgb}{0.55,0.55,0.55}
\definecolor{mPurple2}{rgb}{0.63,0.05,0.05}
\definecolor{backgroundColour}{rgb}{0.95,0.95,0.92}
\definecolor{backgroundColour2}{rgb}{0.95,0.92,0.95}

\lstdefinestyle{C}{
    backgroundcolor=\color{backgroundColour},   
    commentstyle=\color{mGreen},
    keywordstyle=\color{blue},
    numberstyle=\tiny\color{mGray},
    stringstyle=\color{mPurple},    
    basicstyle=\footnotesize,
    breakatwhitespace=false,         
    breaklines=true,                 
    captionpos=b,                    
    keepspaces=true,                 
    numbers=left,                    
    numbersep=5pt,                  
    showspaces=false,                
    showstringspaces=false,
    showtabs=false,                  
    tabsize=2,
    language=C
}

\definecolor{t_comment}{rgb}{0.2,1,0.2}
\definecolor{t_mGray}{rgb}{0.5,0.5,0.5}
\definecolor{t_mPurple}{rgb}{0.58,0,0.05}
\definecolor{t_blue}{rgb}{0.4,0.6,0.8}
\definecolor{t_mGreen2}{rgb}{0.05,0.65,0.05}
\definecolor{t_mGray2}{rgb}{0.75,0.75,0.75}
\definecolor{t_mPurple2}{rgb}{0.63,0.05,0.05}
\definecolor{t_bg}{rgb}{0.15,0.15,0.18}

\lstdefinestyle{terminal}{
    backgroundcolor=\color{t_bg},   
    commentstyle=\color{t_comment},
    keywordstyle=\color{t_blue},
    numberstyle=\tiny\color{t_mGray},
    stringstyle=\color{t_mGray2}, 
    basicstyle=\footnotesize\color{t_mGray2},
    breakatwhitespace=false,         
    breaklines=true,                 
    captionpos=b,                    
    keepspaces=true,                 
    numbers=none,                    
    numbersep=5pt,                  
    showspaces=false,                
    showstringspaces=false,
    showtabs=false,                  
    tabsize=2,
    language=C
}

\graphicspath{{.}}
\begin{document}

\begin{center}
\fbox{\fbox{\parbox{5.5in}{\centering
\Large{COMPSCI 3MI3 : Assignment 5} \\
\large{Fall 2021} \\ 
Nicholas Moore}}}
\end{center}

\begin{questions}
\question[8] \textbf{Fibonacci Sequence} \\
The Fibonacci numbers can be defined as follows:

\begin{equation}
Fib(n) = \begin{cases}
0 & n = 0 \\
1 & n = 1 \\
Fib(n-1) + Fib(n-2) & n >= 2 
\end{cases}
\end{equation}

Create a $\lambda$ expression in the enriched $\lambda$-Calculus which, when input to the Fixed Point Combinator, takes in a position in the Fibonacci sequence, and outputs the corresponding Fibonacci number.  \\

Demonstrate your function works by evaluating the $4^{th}$ number in the sequence.  

\question[6] \textbf{Determinacy of $\lambda$-Calculus} \\
In UAE, we discussed the property of determinacy at some length.  Is the call-by-value evaluation strategy of $\lambda$-Calculus determinate?  If yes, provide a (traditional) proof of determinacy.  If no, provide a (traditional) proof of non-determinacy. \\

The operational semantics of the call-by-value strategy are given below. \\
\includegraphics[scale=0.35]{semantics.png}

\question[6] \textbf{Termination of $\lambda$-Calculus} \\ 
A language is said to terminate if a finite set of terms will always result in a finite evaluation chain.  This property does not hold in $\lambda$-Calculus, but no rigorous proof of this fact was provided in lecture.  \\

Create a rigorous (traditional) proof that the Termination property does not hold for $\lambda$-Calculus.  \\

HINT: Examining the proof of termination for UAE, as well as the section from slide set 6 on computability may help you structure this proof.  

\end{questions}
\end{document}