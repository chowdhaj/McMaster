\documentclass{exam}
\usepackage[utf8]{inputenc}
\usepackage{amsmath}
\usepackage{listings}
\usepackage{graphicx}
\usepackage{hyperref}
\usepackage{xcolor}
\usepackage{listings}

\let\OldTexttt\texttt

\renewcommand{\texttt}[1]{\OldTexttt{\color{teal}{#1}}}
\definecolor{mGreen}{rgb}{0,0.6,0}
\definecolor{mGray}{rgb}{0.5,0.5,0.5}
\definecolor{mPurple}{rgb}{0.58,0,0.05}
\definecolor{mGreen2}{rgb}{0.05,0.65,0.05}
\definecolor{mGray2}{rgb}{0.55,0.55,0.55}
\definecolor{mPurple2}{rgb}{0.63,0.05,0.05}
\definecolor{backgroundColour}{rgb}{0.95,0.95,0.92}
\definecolor{backgroundColour2}{rgb}{0.95,0.92,0.95}

\lstdefinestyle{C}{
    backgroundcolor=\color{backgroundColour},   
    commentstyle=\color{mGreen},
    keywordstyle=\color{blue},
    numberstyle=\tiny\color{mGray},
    stringstyle=\color{mPurple},    
    basicstyle=\footnotesize,
    breakatwhitespace=false,         
    breaklines=true,                 
    captionpos=b,                    
    keepspaces=true,                 
    numbers=left,                    
    numbersep=5pt,                  
    showspaces=false,                
    showstringspaces=false,
    showtabs=false,                  
    tabsize=2,
    language=C
}

\definecolor{t_comment}{rgb}{0.2,1,0.2}
\definecolor{t_mGray}{rgb}{0.5,0.5,0.5}
\definecolor{t_mPurple}{rgb}{0.58,0,0.05}
\definecolor{t_blue}{rgb}{0.4,0.6,0.8}
\definecolor{t_mGreen2}{rgb}{0.05,0.65,0.05}
\definecolor{t_mGray2}{rgb}{0.75,0.75,0.75}
\definecolor{t_mPurple2}{rgb}{0.63,0.05,0.05}
\definecolor{t_bg}{rgb}{0.15,0.15,0.18}

\lstdefinestyle{terminal}{
    backgroundcolor=\color{t_bg},   
    commentstyle=\color{t_comment},
    keywordstyle=\color{t_blue},
    numberstyle=\tiny\color{t_mGray},
    stringstyle=\color{t_mGray2}, 
    basicstyle=\footnotesize\color{t_mGray2},
    breakatwhitespace=false,         
    breaklines=true,                 
    captionpos=b,                    
    keepspaces=true,                 
    numbers=none,                    
    numbersep=5pt,                  
    showspaces=false,                
    showstringspaces=false,
    showtabs=false,                  
    tabsize=2,
    language=C
}

\graphicspath{{.}}
\begin{document}

\begin{center}
\fbox{\fbox{\parbox{5.5in}{\centering
\Large{COMPSCI 3MI3 : Assignment 8} \\
\large{Fall 2021} \\ 
Nicholas Moore}}}
\end{center}

\begin{questions}
\question \textbf{Proof of Sequencing as a Derived Form} \\ 
In topic 9, slides 26-32, we discuss the sequencing operator \texttt{;} in two ways, as a separate term of the language, and as a term derived from an inner language.  In these slides, we stated the following theorem. \\ 

\textbf{THEOREM [Sequencing is a Derived Form]} \\
Define $\lambda^\mathcal{E}$ as the \textbf{external calculus}.  This language will be composed of simply typed $\lambda$-Calculus, enriched with the Unit type and term, and with the term ($t_1 ; t_2$), E-Seq, E-SeqNext, and T-Seq. \\

Define $\lambda^\mathcal{I}$ as the \textbf{internal calculus}.  This language will be composed of \emph{only} the simply typed $\lambda$-Calculus and Unit type and term. \\

Define $e \in \lambda^\mathcal{E} \rightarrow \lambda^\mathcal{I}$ as an \textbf{elaboration function}, which translates from the external language to the internal language.  It does so by replacing all instances of $t_1 ; t_2$ with $(\lambda x : Unit.t_2)\:t_1$. For each term $t$ of $\lambda^\mathcal{E}$, we have:
\begin{equation}
t \xrightarrow{\mathcal{E}} t' \iff e(t) \xrightarrow{\mathcal{I}} e(t')
\end{equation}
\begin{equation}
\Gamma \vdash^{\mathcal{E}} t : T \iff \Gamma \vdash^{\mathcal{I}} e(t) : T
\end{equation}

The proof of these statements proceeds by structural induction over $t$.  

Because these are ``if and only if'' statements, \emph{both} directions must be proven independently.  

\begin{parts}
\part[6] Prove $t \xrightarrow{\mathcal{E}} t' \implies e(t) \xrightarrow{\mathcal{I}} e(t')$
\part[6] Prove $e(t) \xrightarrow{\mathcal{I}} e(t') \implies t \xrightarrow{\mathcal{E}} t' $
\part[6] Prove $\Gamma \vdash^{\mathcal{E}} t : T \implies \Gamma \vdash^{\mathcal{I}} e(t) : T$
\part[6] Prove $\Gamma \vdash^{\mathcal{I}} e(t) : T \implies \Gamma \vdash^{\mathcal{E}} t : T$
\end{parts}

\end{questions}
\end{document}